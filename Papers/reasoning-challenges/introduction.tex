\section{Introduction}
\label{sec:introduction}


The International Planning Competition (IPC)\footnote{See
\url{http://ipc.icaps-conference.org/}.} recently celebrated its 10-year
anniversary in 2008. During this time, six instances of the IPC have been
run, each of which has helped drive the development of new planning
technology, by providing a regular forum for showcasing the latest trends
in planner design, domain description languages, and challenging planning
domains. Given the absence of a competition in 2010---breaking the
competition's established two year cycle for the first time since its
inception---we feel the time is right to revisit the structure of the
existing IPC, and suggest a new direction for the competition in the
future.

This paper proposes a new direction for the IPC, building on the strength
of previous competitions, but centred around a novel online infrastructure.
Most notably, we envision a more interactive and potentially more
wide-ranging competition featuring:
%
\begin{itemize}
\item A ``rolling'' competition format with community-initiated track and
problem specifications,

\item Continuous evaluation of contributed planners on competition
problem sets, with real-time access to performance data generated on a
common hardware platform, and

\item A centralised repository offering researchers access to the latest
domains and planners, and a forum for community news, discussion, and
interaction.
\end{itemize}

Thus, while previous instances of the IPC have traditionally been
event-based competitions, scheduled at regular intervals, we propose an
ongoing competition as a permanent fixture for the planning community. In
particular, members of the community would be able to upload new planners
and problem domains to a central competition server that continually
re-evaluates planners and generates performance data. Up-to-the-minute
results would be centrally available, while more formal competitions could
be ``run'' at regular intervals by producing snapshots of current planners
and problems. 

From a technical point of view, it is important to note that such an idea
would not have been practical ten years ago during the time of the first
IPC. It is only recently, due to advances in computing hardware and network
bandwidth, that a scenario such as this is now feasible. Even so, our
proposed infrastructure nevertheless presents some technical challenges
that must be addressed (e.g., security, , accessibility, maintenance,
etc.). However, we believe that for the most part, solutions to these
problems can already be found by looking at similar (although smaller
scale) competition efforts in the wider research community.

Most importantly, we believe this proposal offers significant benefits to
competition organisers, participants, and the wider planning community.
\textbf{[RP: say something more here]} Furthermore, since much of the
infrastructure we describe here is not tied to a specific research area,
attempts to establish similar competitions could benefit from this
approach, leading to joint ventures with other research fields---and the
possibility of future inter-disciplinary challenges.

In the remainder of the paper we outline our proposal for a new competition
infrastructure, and the benefits we believe this approach could bring to
the planning community and beyond.

