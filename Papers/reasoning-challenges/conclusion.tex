\section{Conclusion}
\label{sec:conclusion}

\todo{this is cut and paste from Guido's text, which I like but which
  was in a different place. The conclusion should be cleaned up and
  made real text. - ak}

This new approach has advantages for each of the three groups of
people involved in a competition, the organisers, the participants,
and the audience (which usually includes organisers and
participants). \todo{do we want to use the term ``stakeholders'' for
  all of these people together? - ak}

\begin{itemize}
  \item Organisers benefit from a mostly automated platform that is maintained by the community, which should greatly reduce the effort needed to run a competition.
  \item Participants benefit from a well-defined environment for running their systems, short feedback times after submitting a system, and a standardised test and benchmark platform for experimenting with novel algorithms.
  \item The audience, i.e., the scientific community, benefits from the improved discussion and report facilities. The scientific value of competitions is to find out \emph{why} a system shows a certain performance, not only \emph{that} it does.
\end{itemize}



\todo{I don't think we should have acknowledgments in the anonymous
  version. - ak}

\subsubsection*{Acknowledgements}
The original idea to organize a ``rolling'' competition in the CP community is due to Christian Schulte and Peter Stuckey.




%%% Local Variables: 
%%% mode: latex
%%% TeX-master: "main"
%%% End: 
