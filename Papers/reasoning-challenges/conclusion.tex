\section{Conclusion}
\label{sec:conclusion}

%\todo{this is cut and paste from Guido's text, which I like but which
%  was in a different place. The conclusion should be cleaned up and
%  made real text. - ak}

The new direction we propose for the IPC has significant advantages for
each of the competition's three main groups of stakeholders: the
organisers, the participants, and the audience.  Organisers benefit from a
mostly automated platform that is maintained by the community, which should
greatly reduce the effort needed to run a competition. Participants benefit
from a well-defined environment for running their systems, short feedback
times for system evaluation, and a standardised test and benchmark platform
for experimenting with novel algorithms. The audience -- in this case the
wider planning and scientific communities -- benefits from improved
discussion and reporting facilities. Since the scientific value of
competitions is to find out \emph{why} a system performs the way it does,
not only \emph{that} it performs a certain way, enabling users to
contribute to the competition process reduces the chance that essential
avenues of investigation and evaluation are missed. Instead, a robust and
active competition community only serves to enhance the wider planning
community as a whole.


%This new approach has advantages for each of the three groups of
%people involved in a competition, the organisers, the participants,
%and the audience (which usually includes organisers and
%participants). \todo{do we want to use the term ``stakeholders'' for
%  all of these people together? - ak} %RP: I like stakeholders.

  
  

  
  




%\todo{I don't think we should have acknowledgments in the anonymous
%  version. - ak}
%RP: I agree; the acknowledgements should be left out to anonymise the
%paper

%\subsubsection*{Acknowledgements}
%The original idea to organize a ``rolling'' competition in the CP
%community is due to Christian Schulte and Peter Stuckey.




%%% Local Variables: 
%%% mode: latex
%%% TeX-master: "main"
%%% End: 
