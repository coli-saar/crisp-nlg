
\begin{abstract}
The International Planning Competition (IPC) recently celebrated its
10-year anniversary in 2008. During this time, the IPC has helped drive the
development of new planning technology, by providing a regular forum for
showcasing the latest trends in planner design, domain description
languages, and challenging planning domains. Given the absence of a
competition in 2010, we feel the time is right to revisit the structure of
the existing IPC. Building on the strengths of previous competitions, we
propose a new type of ``rolling'' planning competition, as a permanent
fixture for the planning community. A central competition server would
continually evaluate planners and generate performance data, while members
of the community could directly upload new planners and problem domains,
with the server automatically reevaluating existing planners as necessary.
Real-time results would continuously be available, while more formal
competitions could be ``run'' at regular intervals by producing snapshots
of current planners and problems. Such a proposal offers significant
benefits: researchers would have access to a central repository offering
the latest domains and planner performance statistics, running on a common
hardware platform, plus a facility for related activities such as posting
challenges and demonstrating new technologies. Since much of the proposed
infrastructure is not tied to a specific research area, we believe such an
approach also offers the possibility of joint ventures with other
interested research communities.
\end{abstract}

