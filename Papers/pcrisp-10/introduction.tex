\section{Introduction}

Many sentence generation systems are organized in a pipeline
architecture, in which the input semantic representation is first
enriched, e.g.\ with referring expressions, by a \emph{sentence
  planner} and only then transformed into natural language strings by
a \emph{surface realizer} \cite{reiterdale2000}.  An alternative
approach is \emph{integrated} sentence generation, in which both steps
are performed by the same algorithm, as in the SPUD system
\cite{Stone2003a}.  An integrated algorithm can sometimes generate
better and more succinct sentences \cite{stone98textual}.  SPUD itself
gives up some of this advantage by using a greedy search heuristic
for efficiency reasons.  The CRISP system, a recent reimplementation
of SPUD using search techniques from AI planning, achieves high
efficiency without sacrificing complete search
\cite{kollerstone2007,KolHof10}.

While CRISP is efficient enough to perform well on large-scale
grammars \cite{KolHof10}, such grammars tend to offer many
different ways to express the same semantic representation.  This
makes it necessary for the generation system to be able to compute not
just grammatical sentences, but to identify which of these sentences
are \emph{good}. This problem is exacerbated when using
treebank-derived grammars, which tend to underspecify the actual
constraints on grammaticality and instead rely on statistical
information learned from the treebank.  Indeed, there have been a
number of systems for statistical generation, which can exploit such
information to rank sentences appropriately
\cite{langkildeknight1998,whitebaldridge2003,belz2008}.  However, to our 
knowledge, all such systems are currently restricted to performing surface realization,
and must rely on separate modules to perform sentence planning.

In this paper, we bring these two strands of research together for the
first time.  We present the PCRISP system, which redefines the SPUD
generation problem in terms of probabilistic TAG grammars (PTAG,
\cite{resnik1992}) and then extends CRISP to solving the probabilistic
SPUD generation problem using metric planning \cite{fox2002,hoffmann2003}.  We
evaluate PCRISP on a PTAG treebank extracted from the Wall Street
Journal Corpus \cite{chenschanker2004}.  The evaluation reveals
a tradeoff between coverage, efficiency, and accuracy which we think
are worth exploring further in future work.

\paragraph{Plan of the paper.} We start by putting our research in the
context of related work in Section~\ref{sec:related} and reviewing
CRISP in Section~\ref{sec:crisp}.  We then describe PCRISP, our
probabilistic extension of CRISP, in Section~\ref{sec:pcrisp} and
evaluate it in Section~\ref{sec:experiments}. We conclude in
Section~\ref{sec:conclusion}. 



%%% Local Variables: 
%%% mode: latex
%%% TeX-master: "pcrisp-10"
%%% End: 
