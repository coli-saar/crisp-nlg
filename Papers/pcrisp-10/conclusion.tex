\section{Conclusion}
\label{sec:conclusion}

We have described PCRISP, an approach to integrated sentence generation which can compute the best derivation according to a probabilistic TAG grammar.  This brings two strands of research -- statistical generation and integrated sentence planning and realization -- together for the first time.  Our generation algorithm operates by converting the generation problem into a metric planning problem and solving it with an off-the-shelf planner.  An evaluation on the WSJ corpus reveals that PCRISP, like PTAG in general, is susceptible to data sparseness problems.  Because the size of the planning problem is quadratic in the number of lexicalized trees in the grammar, current planning algorithms are also too slow to be used for longer sentences.

An obvious issue for future research is to apply improved smoothing techniques to deal with the data sparseness.  Planning runtimes should be improved by further tweaking the exact planning problems we generate, and will benefit from any future improvements in metric planning.  It is interesting to note that the extensions we made to CRISP to accommodate statistical generation here are compatible with recent work in which CRISP is applied to situated generation \cite{garoufikoller2010}; we expect that this will be true for other future extensions to CRISP as well.  Finally, we have only evaluated PCRISP on a surface realization problem in this paper.  It would be interesting to carry out an extrinsic, task-based evaluation of PCRISP that also addresses sentence planning.

\paragraph{Acknowledgments.} We are grateful to Owen Rambow for providing us with the Chen WSJ-TAG corpus and to Malte Helmert and Silvia Richter for their help with running LAMA, another metric planner with which we experimented. We thank Konstantina Garoufi and Owen Rambow for helpful discussions, and our reviewers for their insightful comments.




%%% Local Variables:  
%%% mode: latex 
%%% TeX-master: "pcrisp-10" 
%%% End: 


