\begin{abstract}
We present a planning domain that encodes the problem of generating
natural language sentences. This domain has a number of features that
provoke fairly unusual behavior in planners. In particular, hitherto
no existing automated planner was sufficiently effective to be of
practical value in this application. We analyze in detail the reasons
for ineffectiveness in FF, resulting in a few minor implementation
fixes in FF's preprocessor, and in a basic reconfiguration of its
search options. The performance of the modified FF is up to several
orders of magnitude better than that of the original FF, and for the
first time makes automated planners a practical possibility for this
application. Beside thus highlighting the importance of preprocessing
and automated configuration techniques, we show that the domain still
poses several interesting challenges to the development of search
heuristics.
%In particular, we show how several techniques used in the FF planner,
%which has been highly successful on IPC problems, are led astray. We
%discuss possible fixes to FF's shortcomings, some of which we have
%already implemented, but some of which pose interesting challenges to
%the AI Planning community. The performance of our modified FF is up to
%several orders of magnitude better than that of the original FF, and
%for the first time makes automated planners a pratical possibility for
%this application.
%
%. While this makes FF
%much more practical for this application, we show that it is still
%quite limited with respect to the complexity of the underlying
%grammar.
%We outline how other planning domains might share the
%features that make the generation domain difficult. {\bf do we? how?}
\end{abstract}


%%% Local Variables: 
%%% mode: latex
%%% TeX-master: "main"
%%% End: 
