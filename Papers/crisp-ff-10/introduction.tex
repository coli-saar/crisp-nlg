\section{Introduction} \label{sec:introduction}

Natural language generation (NLG) -- the problem of computing a
sentence or text that communicates a given piece of information -- has
a long-standing connection to automated planning
\cite{perrault80,appelt:planning}. In the past few years, the idea has
been picked up again in the NLG community, both to use planning as an
approach for modeling language problems
\cite{Steedman-Petrick:07,benotti08b} and to tackle the search
problems in existing NLG problems by using techniques from planning
\cite{KolSto07}.

Existing planners are relatively good at solving modest instances of
the planning problems that arise in this context. However, Koller and
Petrick \shortcite{koller10:_exper_with_plann_for_natur_languag_gener}
recently showed that as the planning problem instances from Koller and
Stone \shortcite{KolSto07} scale up, off-the-shelf planners become too
slow to be practically useful. They noted that there are major sources
of inefficiency both in the search itself and in the preprocessor,
which is put under pressure by the sheer number of ground instances of
facts and actions that come up in this domain.

In this paper, we revisit this situation, analyze the sources of
inefficiency in more detail in the context of the FF planner
\cite{HoffmannNebel01}, and show how FF can be optimized to perform
dramatically better -- and in fact, well enough to be useful in
practical NLG applications.  We achieve this by tweaking the
implementation of the FF preprocessor and analyzing the behavior of
the enforced hillclimbing and best-first search strategies on the NLG
domain.  Neither of these changes degrades FF's performance on other
domains.

Nevertheless, we can still generate realistic instances of the NLG
planning domain on which our modified FF becomes too slow for
practical use.  As FF is still a competitive planner, this suggests
that NLG might be interesting as a future challenge domain.  In
particular, we show that by modifying the linguistic grammar from
which we generate, we can vary the planning problems quite widely and
emphasize different aspects that make our domain hard.



%%% Local Variables: 
%%% mode: latex
%%% TeX-master: "main"
%%% End: 
