
\section{Introduction} \label{sec:introduction}

Natural language generation (NLG) -- the problem of computing a
sentence or text that communicates a given piece of information -- has
a long-standing connection to automated planning \cite{perrault80}. In
the past few years, the idea has been picked up again in the NLG
community, both to use planning as an approach for modeling language
problems (e.g., \cite{Steedman-Petrick:07}) and to tackle the search
problems in NLG by using techniques from planning \cite{KolSto07}.

Existing planners are relatively good at solving modest instances of
the planning problems that arise in this context. However, Koller and
Petrick \shortcite{koller10:_exper_with_plann_for_natur_languag_gener}
recently showed that as the planning problem instances from Koller and
Stone \shortcite{KolSto07} scale up, off-the-shelf planners become
slow -- indeed, too slow to be practically useful in NLG
applications. They showed this for SGPlan and several variants of FF
\cite{HoffmannNebel01}; we show it herein also for LAMA
\cite{richter:etal:aaai-08}. Koller and Petrick noted that there are
major sources of inefficiency both in the search itself and in the
commonly used preprocesses; they identified several features of the
domain (e.g.\ comparatively large predicate and operator arities) that
diverge from most IPC benchmarks.
%the sheer number of
%ground instances of facts and actions that come up in this domain.

In this paper, we revisit this situation, and analyze the sources of
inefficiency in detail for FF. Based on the analysis, we modify FF and
obtain dramatically better performance that, for the first time, makes
automated planners a realistic possibility for NLG applications. To
achieve this quantum leap, all that is needed are a few minor
implementation fixes and basic reconfiguration of search
options. While these fixes certainly are no contribution to planning
in their own right, we consider it a valuable -- and slightly
unsettling -- lesson learned that such banalities can make the
difference between success and failure. They can hold up progress in
planning applications for years, unless users of a planner have direct
access to its developer. We interpret this as illustrative of the
importance of implementation details such as the preprocessor for the
overall runtime, and as a strong call for automated configuration
techniques, e.g.\ along the lines of Hutter et al.\
\shortcite{HutHooStu07}.


Apart from these insights, our domain remains an interesting challenge
for the development of better heuristics. There still are realistic
instances of the NLG planning domain on which our modified FF is not
practical. Overcoming this weakness would entail overcoming
fundamental weaknesses of FF's goal ordering techniques and the
relaxed plan heuristic -- variants of which are used in many planners
today.



% show how to modify FF to obtain dramatically better performance. We
% achieve this by tweaking the implementation of FF's preprocessor, and
% by acting on some simple but shocking observations regarding the
% behavior of some of FF's search techniques, in particular its goal
% ordering technique and the relaxed plan heuristic. The changes we
% make are all simple, but make FF perform well enough to be useful in
% practical NLG applications.

%the enforced hillclimbing and best-first search strategies on the NLG
%domain.  Neither of these changes degrades FF's performance on other
%domains.

%Nevertheless

%Our performance gains with simple FF configuration changes certainly
%point out that automatic planner configuration may be quite
%useful. Apart from that, our domain remains an interesting challenge
%for the development of better heuristics. There still are realistic
%instances of the NLG planning domain on which our modified FF is not
%practical. Overcoming this weakness would entail overcoming the
%observed weaknesses of FF's search techniques -- variants of which are
%used in many planners today.


%As FF is still a competitive
%planner, this suggests that NLG might be interesting as a future
%challenge domain. 
%
% In particular, we show that by modifying the
%linguistic grammar from which we generate, we can vary the planning
%problems quite widely and emphasize different aspects that make our
%domain hard.



%%% Local Variables: 
%%% mode: latex
%%% TeX-master: "main"
%%% End: 
