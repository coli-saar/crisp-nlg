
\section{Conclusion} \label{sec:conclusion}


Generation of natural language sentences is interesting as a planning
problem, both in its application value and in the challenges it
poses. After the changes to FF described in this paper, for the first
time an automated planner is a practical option for this
application. The simplicity of the changes required is striking, and
strongly suggests to look more closely at automatic planner
configuration techniques.

%It is striking how simple these changes were, and certainly this
%suggests to look more closely at automatic planner configuration
%techniques.

The domain is far from ``solved'' in all its aspects. As we pointed
out, several features of the domain are very hard to capture with
existing search heuristics. In Fig.~\ref{fig:runtimes}, this shows in
the fact that no planner version scales beyond $n>4$ -- even in case
(a) where no adjectives at all are required. It remains an interesting
challenge to address this.


\section*{Acknowledgements}
We thank Konstantina Garoufi for driving sentence generation with FF
to its limits, Ron Petrick for fruitful discussions, Silvia Richter
for making LAMA available, and the reviewers for their helpful
comments.
%\todo{Wollen wir noch jemandem danken?}

%While this brings the overall planner runtimes into a range where FF
%can now be useful in practical NLG applications, it is important to
%note that it still has serious limitations.  The best-first search
%timed out for any $n>4$ in either experiment.  While many NLG
%applications can get by with shorter sentences, improving the search
%on our domain is still an open question for the future.  Even within
%our domain, the relative quality of EHC and BFS+H depended on the
%exact generation problem instance: While BFS+H outperformed EHC at
%$d=2$, EHC was actually slightly faster at $d=0$ because the absence
%of distractors meant there was not much room for the characteristic
%underestimation of the relaxed plan evaluation discussed above.  This
%means that sentence generation remains as a varied and challenging
%domain for planning.



%%% Local Variables: 
%%% mode: latex
%%% TeX-master: "main"
%%% End: 
