
\section{Conclusion} \label{sec:conclusion}


Generation of natural language sentences is interesting, as a planning
problem, both in its application value and in the challenges it
poses. Our observations herein have above all the message that, as of
now, FF is a practical option for this application. It is baffling
with which ease this milestone came about, and certainly this suggests
to look more closely at automatic planner configuration techniques.

The domain is also far from ``solved'' in all its aspects. As we
pointed out, several features of the domain are very hard to capture
with existing search heuristics. In Fig.~\ref{fig:runtimes}, this
shows in the fact that no planner version scales beyond $n>4$ -- even
in case (a) where no adjectives at all are required. It remains an
interesting challenge to address this.



%While this brings the overall planner runtimes into a range where FF
%can now be useful in practical NLG applications, it is important to
%note that it still has serious limitations.  The best-first search
%timed out for any $n>4$ in either experiment.  While many NLG
%applications can get by with shorter sentences, improving the search
%on our domain is still an open question for the future.  Even within
%our domain, the relative quality of EHC and BFS+H depended on the
%exact generation problem instance: While BFS+H outperformed EHC at
%$d=2$, EHC was actually slightly faster at $d=0$ because the absence
%of distractors meant there was not much room for the characteristic
%underestimation of the relaxed plan evaluation discussed above.  This
%means that sentence generation remains as a varied and challenging
%domain for planning.



%%% Local Variables: 
%%% mode: latex
%%% TeX-master: "main"
%%% End: 
