\section{FF Pitfalls and Fixes}
\label{sec:crisp-ff}

%FF exhibits a number of problems in this domain that were previously
%unknown.  

The basic observation of FF's inefficiency in the NLG domain was made
before \cite{koller10:_exper_with_plann_for_natur_languag_gener}; here
we examine its causes in detail. For some of the observed pitfalls, we
are able to provide simple fixes. Others pose serious challenges to
planning research.


\subsection{Preprocessing}
\label{sec:crisp-ff:preprocess}



%../ff -o xtag-3-2-domain.lisp -f xtag-3-2-problem.lisp -T -H: 0.10
%../ff -o xtag-3-2-domain.lisp -f xtag-3-2-problem.lisp -T: 1.54
%../ff -o xtag-3-2-domain.lisp -f xtag-3-2-problem.lisp -H: 0.30
%../ff -o xtag-3-2-domain.lisp -f xtag-3-2-problem.lisp: 1.80


%../ff -o xtag-5-2-domain.lisp -f xtag-5-2-problem.lisp -T -H: 0.29
%../ff -o xtag-5-2-domain.lisp -f xtag-5-2-problem.lisp -T: seg fault ie too much memory
%../ff -o xtag-5-2-domain.lisp -f xtag-5-2-problem.lisp -H: 11.08
%../ff -o xtag-5-2-domain.lisp -f xtag-5-2-problem.lisp: seg fault ie too much memory


%../ff -o xtag-3-3-domain.lisp -f xtag-3-3-problem.lisp -T -H: 
%../ff -o xtag-3-3-domain.lisp -f xtag-3-3-problem.lisp -T: seg fault ie too much memory
%../ff -o xtag-3-3-domain.lisp -f xtag-3-3-problem.lisp -H: 
%../ff -o xtag-3-3-domain.lisp -f xtag-3-3-problem.lisp: seg fault ie too much memory








%FF's preprocessor contains two details that, before this paper, were
%so detrimental that we were not able to seriously propose FF as a
%practical solution.
%Simple fixes
%to these details are so effective that FF is now a practical (if not
%perfect) possibility.



While preprocessing the ADL formulas, FF has a subroutine that
checks the grounded formulas
% -- quantifiers already compiled into
%grounded conjunctions and disjunctions -- 
for tautologies. This involves a loop checking all pairs of
sub-formulas within every conjunction and disjunction, testing whether
they are identical/identical modulo their sign. For example, if we
find $\phi$ and $\neg \phi$ within a disjunction, then the disjunction
is replaced by TRUE. We haven't tested whether this subroutine has a
notable effect in other benchmarks. In our domain, the only effect it
has is to take a huge amount of runtime on the long goal conjunctions
stating that we do not want any open substitution nodes or
distractors. The fix is to switch this subroutine off. This trivial
operation alone can render previously infeasible instances
(preprocessing time $>15$ minutes) harmless (instance solved in $6$
seconds {\em total} time).

%In small instances, this gives advantages like
%$0.29s$ vs.\ $11s$ pre-processing time. In a larger instance we ran,
%original FF's pre-process doesn't finish within 15 minutes whereas our
%fix yields a pre-processing time of about $3s$.


FF's preprocessor contains a very small imprudence that apparently
didn't surface in other domains, but that does in ours. FF
distinguishes operators into ``easy'' (DNF precondition) and ``hard''
(non-DNF precondition), which go through different instantiation
procedures. The procedure for hard operators uses a data structure
encompassing an integer for every possible instantiation of all
predicates. Since this does not take into account reachability, it is
prohibitively huge in our domain. Now, we actually have no hard
operators. But FF builds the data structure anyway, which exhausts
memory and causes a segmentation fault on serious instances. 

%The
%obvious fix is to not build the data structure if there are no hard
%operators.

%We fix this problem by only computing the data structure for problem
%instances that actually contain hard operators.




















\subsection{Search Techniques}
\label{sec:crisp-ff:preprocess:heuristics}





FF's preprocessing pitfalls are certainly baffling, but easily
fixed. We now turn our attention to some of FF's search techniques,
variations of which are used in many planners, and which are very much
not easily fixed. Our solution for the moment are simple configuration
changes. %; these give much better performance but do .


%\item Goal agenda: For some reason that I don't understand in detail,
%  the goal agenda heuristic chooses that the ``subst'' goal must be
%  reached before even the ``expressed'' goal. This leads the planner
%  to generating some arbitrary sentence. Because the example grammar
%  is so small, this leads to a state from which the ``expressed'' goal
%  can no longer be achieved. (A realistic big grammar should be able
%  to recover from this by introducing a conjunction or relative clause
%  or some such -- but of course, this would still lead to a stupid
%  sentence.) FIX: turn goal agenda off. alternatively, develop
%  appropriate goal ordering techniques.


First, we consider Koehler and Hoffmann's
\shortcite{koehler:hoffmann:jair-00} ``reasonable goal orderings'',
variants and extensions of which are in prominent use today, e.g.\ in
LAMA \cite{richter:etal:aaai-08}. FF as used in the 2000 competition
approximates reasonable orders as a preprocess and then partitions the
goal set into a {\em goal agenda}, a sequence of goal subsets. Each
subset is posed to the planner in isolation. The final plan is the
concatenation of the plans for all entries.

%EXAMPLE missing

The goal agenda is completely off-target in our domain. The most
striking point concerns the interaction between the goal to create a
sentence ($\neg \mathsf{subst}(S,\mathsf{root})$ in the rabbit
example), and the goal to express a certain meaning
($\f{expressed}(\f{sleep},e,r_1)$). The goal agenda orders the former
before the latter because it detects that, once a meaning is
expressed, the sentence creation has started and it is not possible to
start anew anymore. Unless the inverse also holds,\footnote{The
  inverse holds if, in the given grammar, the only way to communicate
  a meaning is in the main clause -- i.e., there are no subordinate
  clauses like ``the sleeping rabbit''.}
$\mathsf{subst}(S,\mathsf{root})$ is alone in the first goal agenda
entry. Consequently, the planner decides to generate {\em any}
sentence, with some arbitrary meaning. If it happens to be the wrong
sentence, the planner is stuck in a dead-end. %
%While that dead end is recognized by
%the relaxed plan heuristic -- i.e., it is easy to see that the actions
%needed for achieving the goal cannot be applied anymore -- certainly
%this is an entirely unintended (and a bit amusing) outcome of this
%technique.
% Switching it off, we {\bf \dots
%example before/after.}
It is not clear to us, at this point, how goal ordering techniques
could be modified so that they return more sensible results.

%The issue regarding $\neg
%\mathsf{subst}(S,\mathsf{root})$ and $\mathsf{sleep}(e,r_1)$ could
%potentially be addressed by recognizing that, once the latter is
%achieved, the former is already true anyway. Whether such techniques
%suffice to obtain useful goal orderings overall remains to be seen.

%More generally however,
%as outlined below the life-cycle of substitution nodes and distractors
%is weird; %
% -- as we
%outline below, it follows a false-true-false pattern --
%it seems difficult to detect this based on domain analysis.










%\item Subst and distractor atoms have a weird lifecycle: They start as
%  false, then get made true at some point, and then false again.
%  After they have been made false, they can never become true
%  again. FIX: no fix just yet. highly non-trivial challenge to
%  generation of h fns. {\bf (QUESTION: does lama do better? do any of
%    the known admissible heuristics do better?)}


In our improved FF, the goal agenda is simply switched off. Having
thus disposed of goal orderings, let us turn our attention to the
relaxed plan heuristic. %
%While the goal agenda meltdown may be more amusing than alarming, we
%next report that relaxed plan heuristics are not far from such a
%meltdown either. They are fatally 
This meets its Waterloo in the unusual life-cycle of facts encoding
substitution nodes and distractors.
%In instances where these
%constructs are important for the construction of a valid sentence,
These facts are required to be false in the goal, and most of them are
initially false. However, any valid plan must make many of them true
at some intermediate time point (e.g.\ distractors arise depending on
the decisions taken by the planner). Hence, from the perspective of
the relaxed plan heuristic, from one state to the next whole classes
of goals suddenly appear ``out of the blue''.
%These goals are not
%accounted for at all in the preceding state.
%, and hence the relaxed plan is
%completely un-informative.


Consider again the rabbit example. In the initial state, the relaxed
plan has length $1$ because we only need to achieve $\neg
\mathsf{subst}(S,\mathsf{root})$ and $\f{expressed}(\f{sleep},e,r_1)$,
which is done by the single action ``sleeps''. However, that action
has the effect $\forall y. y \neq x_1 \rightarrow
\f{distractor}(u_1,y)$, which introduces new distractors. Thus the
relaxed plan for the second state contains \emph{two} actions. In
general (if there are $n$ rabbits, say), the relaxed plan may become
arbitrarily long.
%To give just one concrete example, say we are in the initial state,
%and have not yet committed to a sentence. For certain kinds of
%sentences ({\bf more concrete here???}), the relaxed plan has length
%$1$ because the we only need to choose the desired meaning. However,
%in the state $s$ resulting from this action, the mechanisms of the
%planning task introduce all the sub-goals that are required in order
%to pin-point that meaning (in particular we get a new goal for all the
%distractors that must be removed). Hence the relaxed plan is much
%longer than before -- arbitrarily longer, in general.  
For FF's particular search algorithm, enforced hill-climbing (EHC),
this means that {\em FF turns into a blind breadth-first search}: EHC
starts at $h$ value $1$ and thereafter needs to solve the whole
problem in order to find a (goal) state with lower $h$ value.

%../ff -o modifiers-10-10-domain.lisp -f modifiers-10-10-problem.lisp -T -H

%\item As a consequence, Enforced hillclimbing: Initial state of
%  example problem has a relaxed graph of length one because
%  ``expressed'' and ``subst'' goals are achieved, and newly introduced
%  violations of ``subst'' and ``distractor'' are not noticed. As a
%  consequence, EHC on the second state spends all of its time in
%  breadth-first search. FIX: don't use EHC.


More generally, from this example we immediately see that, in the
terms of Hoffmann \shortcite{hoffmann:jair-05}, the exit distance from
local minima under $h^+$ is unbounded. Also, we have unrecognized
dead-ends, i.e., dead-end states that have a relaxed plan. An example
is the initial state of a task that it unsolvable because not all
distractors can be removed; its $h$ value will still be $1$. So the
domain is in the most difficult class of Hoffmann's
\shortcite{hoffmann:jair-05} planning domain taxonomy. While it shares
this property with several other challenging domains, turning EHC into
a breadth-first search is unheard of.


%Can we design heuristics that are better at dealing with the lifecycle
%of substitution nodes and distractors? The problem of not noticing
%them in the initial state is present for all approaches that are
%dominated by $h^+$,
%e.g.\ \cite{karpas:domshlak-ijcai-09,helmert:domshlak:icaps-09}. The
%same problem is also present for the inadmissible landmarks heuristic
%\cite{richter:etal:aaai-08}, as well as for $h^{\mathsf{cea}}$
%\cite{helmert:geffner:icaps-08} and its variant $h^{\mathsf{pcc}}$
%\cite{cai:etal:icaps-09}, and hence also for $h^{\mathsf{CG}}$
%\cite{helmert:icaps-04}. As far as known heuristics are concerned,
%hope thus resides mainly in the $h^m$ family and in abstraction
%heuristics. We ran a few tests with an example like the rabbit one,
%except that there are 10 distractors. The Graphplan estimate for the
%initial state is $4$, i.e., $h^2$ is better than $h^+$ here. Still,
%$4$ is far away from the real goal distance $14$. %
%{\bf here I could use
%  differently scaled versions to see whether the distance estimate is
%  constant across the number of distractors; I would think that
%  yes}. 
% .. have tried IPP on NEUES-MODIFIER-SET but get seg faults and shit..
% oh, whatever.
%With merge\&shrink \cite{helmert:etal:icaps07}, we obtained the
%estimate $11$ for the initial state, which is much better. Still, we
%could not find a parameter setting that reduced the search nodes to
%less than $5200$, which is not much better than blind search ($13800$
%nodes). All in all, finding better heuristic functions for this kind
%of problem remains a challenge.


Can we design heuristics that are better at dealing with the life-cycle
of substitution nodes and distractors? It is easy to see that the
problem of not noticing them in the initial state is present for most
known heuristics
\cite{helmert:icaps-04,richter:etal:aaai-08,karpas:domshlak-ijcai-09,helmert:domshlak:icaps-09,cai:etal:icaps-09}. The
two notable exceptions are the $h^m$ family, and abstraction
heuristics. We ran a few tests with a variant of the example problem
involving 10 rabbits.  The Graphplan estimate for the initial state is
$4$, i.e., $h^2$ is better than $h^+$ here (although still far from
the real goal distance of $14$). With merge\&shrink
\cite{helmert:etal:icaps07}, we obtained the estimate $11$ for the
initial state, which is much better. Still the search space with that
latter heuristic remained fairly large ($5200$ nodes vs.\ $13800$
nodes for blind search), and it is not clear how well the technique
will scale to larger instances and instances with a more complex
grammar. Certainly, interesting challenges remain.







%NEUES-MODIFIER-SET$ ../ff -o modifiers-5-5-domain.lisp  -f modifiers-5-5-problem.lisp -T -H: 102
%NEUES-MODIFIER-SET$ ../ff -o modifiers-5-5-domain.lisp  -f modifiers-5-5-problem.lisp -T -H -N: 282
%NEUES-MODIFIER-SET$ ../ff -o modifiers-10-10-domain.lisp  -f modifiers-10-10-problem.lisp -T -H: 3078
%NEUES-MODIFIER-SET$ ../ff -o modifiers-10-10-domain.lisp  -f modifiers-10-10-problem.lisp -T -H -N: 16875
%NEUES-MODIFIER-SET$ ../ff -o modifiers-15-15-domain.lisp  -f modifiers-15-15-problem.lisp -T -H: ?? sehr lange
%NEUES-MODIFIER-SET$ ../ff -o modifiers-15-15-domain.lisp  -f modifiers-15-15-problem.lisp -T -H -N: ?? sehr lange

The only FF technique that appears to be clearly useful here is its
action pruning technique. While the relaxed plan is often much too
short, it appears to always contain at least one of the actions that
actually are useful.
%
%As our experiments data
%will show, it certainly prunes the set of available actions quite
%effectively.
%
%For example, in the
%10-distractor test instance, $16875$ nodes are explored without
%helpful actions pruning; with pruning, this reduces to $3078$. For
%larger examples, the factor between the two numbers tends to become
%larger. 
%
%
%Since EHC is often completely inadequate, and helpful actions are
%quite effective, 
%
Given the above observations, we modified FF to directly run
best-first search with helpful actions (as
% regarding EHC and helpful actions, our
%modified FF directly runs best-first search with helpful actions (as
opposed to original FF, which switches to best-first with{\em out}
helpful actions if EHC fails).


%\item The only thing that works well is helpful actions. However,
%  these didn't get around to be used due to no helpful actions in
%  best-first search: After failure of EHC, FF falls back to best-first
%  search, but doesn't use the helpful actions heuristic in this
%  search. FIX: simply go straight to BFS with helpful actions.








%%% Local Variables: 
%%% mode: latex
%%% TeX-master: "main"
%%% End: 
