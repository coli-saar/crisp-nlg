

\subsection{Initial Definition}
\label{initial-definition}


\joerg{Note: our email text is contained at the end of the tex file as
  a comment}



Given an operator $o$ with parameters $P(o)$, define as the
\emph{parameter-groups} the set $G(o) = \{G_1, \dots, G_n\}$ of
parameter subsets $G_i \subseteq P(o)$ that arises from starting with
$G(o) = \emptyset$, then looping over all predicate $p$ in $o$'s
description and inserting the set of $p$'s parameters as a new group
$G_i$.


A \emph{$k$-split} of $o$ is a set $S(o) = \{S_1, \dots, S_k\}$ of
parameter subsets so that: (1) $\bigcup_{j=1}^k S_j = P(o)$; (2) for
all $1 \leq i \leq n$ there exists $1 \leq j \leq k$ so that $G_i
\subseteq S_j$. The \emph{arity} of a split $S(o)$ is $max_{1\leq j
  \leq k} |S_j|$. A split is \emph{arity-minimizing} if there exists
no other split (for any $k$) that has smaller arity. The
\emph{overlap} of a split $S(o)$ is $|\bigcap_{j=1}^k S_j|$. A split
is \emph{overlap-minimizing} if there exists no other split that has
smaller overlap.
%$S'(o)$ so that $max_{1\leq j \leq k} |S_j| > max_{1\leq j \leq k'}
%  |S'_j|$. A $k$-split $S(o)$ is \emph{overlap-minimizing} if there
%  exists no other split (for any $k$) $S'(o)$ so that
%  $|\bigcap_{j=1}^k S_j| > |\bigcap_{j=1}^{k'} S'_j|$.
\joerg{I'm not sure if overlap-minimzation is meaningful, but
  I thought I can just as well state it. I haven't yet thought at all
  about any possible implications between the two, neither whether
  finding minimal splits is hard. I would guess it's not.}


We observe that, given a split, we can re-write $o$ into $k$ operators
$o_1, \dots, o_k$ so that executing $o$ is equivalent to executing
$o_1, \dots, o_k$ in sequence -- note that this is basically inverting
all the known work on macro-operators... :-)


We introduce new $0$-ary predicates $processing\mbox{-}none$,
$do\mbox{-}o_2, \dots, do\mbox{-}o_k$. We introduce new $1$-are
predicates $arg_1(.), \dots, arg_l(.)$ where $l = |P(o)|$ is the
number of parameters of $o$, and $x_1, \dots, x_l$ is an arbitrary
ordering of $P(o)$. $processing\mbox{-}none$ becomes a precondition of
every operator other than $o$, as well as of $o_1$; it becomes a
delete of $o_1$, and an add effect of $o_k$. $do\mbox{-}o_i$ becomes
an add effect of $o_{i-1}$, a precondition of $o_i$, and a delete
effect of $o_i$. The precondition/add list/delete list of $o_i$
consist: (a) of those predicates $p$ of $o$ whose parameters are a
subset of $S_i$ -- more precisely, we put each predicates into exactly
one (arbitrary) $o_i$ for which this is the case; (b) for any operator
$op_i$ in which parameter $x_j$ appears for the first time in the
sequence $o_1, \dots, o_k$, $arg_j(x_j)$ is inserted into the add
effect of $o_1$, and into the precondition of any $o_g$, $g > i$,
where $o_g$ also uses parameter $x_j$. \joerg{Unless I'm mistaken,
  this simple construction does the job -- the modified planning task
  has ``the same''set of plans, modulo replacing $o_1, \dots, o_k$
  with an according instantiation of $o$ (of course we need to prove
  this formally). Generalizing this to splitting up several operators
  should be straightforward. ... I think there will also be an
  inter-action with predicate-splitting as known in logics: smaller
  predicate arity will yield smaller parameter groups and thus splits
  with smaller arity!}


For example, consider the blocksworld operator ``move(A,B,C)'', whose
precondition is ``on(A,B),clear(A),clear(C)'', whose add is ``on(A,C),
clear(B)'' and whose del is ``on(A,B), clear(C)''. A 2-split is given
by ``A,B'' and ``A,C''. We get the actions ``move1(A,B)'' with pre
``processing-none, on(A,B), clear(A)'', add ``do-move2, clear(B),
arg1(A), arg2(B)'', del ``on(A B), processing-none''; and
``move2(A,C)''with pre ``do-move2, clear(C), arg1(A)'', add ``on(A C),
processing-none, arg3(C)'', del ``clear(C), do-move2''.



\joerg{I note that this construction may run planners into trouble
  because, somewhere along the way $o_1, \dots, o_k$, a precondition
  may not be satisfied, ie there may not exists any continuation of
  the operator $o$ -- an instantiation of the remaining parameters --
  so that the op execution can be completed. The planner then finds
  itself in a dead-end. One may think of giving the option to ``roll
  back'' an execution. .. well dunno if this would help or be
  detrimental. something to keep in mind.}


















%OLD stuff / meta info



\commentout{

%email feedback by alex and carlos, replies by joerg (1 october 2010)

> The encoding of move(A,B,C) in your draft does not work, for two
>reasons. First, nothing in your encoding enforces that between move1
>and move2, only component actions of "move" can be
>executed. Presumably, putdown2 (created by splitting the putdown
>operator) can also be executed whenever OK is false, even if do-move2
>is true.

Nope -- putdown2 has the precondition "do-putdown2" which will not be fulfilled here.

> The second problem is that nothing in the encoding enforces that A
> is bound to the same value in move1 and move2. One could easily
> "split" move(a1,b,c) into the two actions move1(a1,b) and
> move2(a2,c), and the planner would see nothing wrong with it. This
> means that we need a mechanism for keeping track of the variable
> bindings.

>
> We propose the following variant of your mechanism:
>
> move1(A,B)
>  - pre: processing(none), on(A,B), clear(A)
>  - effects: ~processing(none), processing(move), clear(B), ~on(A,B), arg1(A), arg2(B)
>
> move2(A,C)
>  - pre: processing(move), arg1(A), clear(C)
>  - effects: ~processing(move), processing(none), arg3(C), on(A,C)

Yes, good point. I incorporated this into the draft.

> - At each point in the plan, exactly one instance of processing(X)
>is true. While a split of the action "a" is processed, this is the
>atom processing(a); otherwise it is processing(none).

Yep; the same effect is achieved via my construction, where the "do-*"
predicate is true only for one operator (and one part i of that
operator). I did adopt, in the draft, your name "processing-none"
rather than my nondescript "OK".

Note that I say "operator" not "action". This is old-school talk for
distinguishing between the high-level "operators", using variables, in
the usual PDDL file, and the grounded propositional-logic "actions". I
know, it's not very consistent because in PDDL the denotation actually
is "action"... ;-) ... anyway, let's please stick to this.

> If the order in which the component actions of the split are
>  executed is important, one could keep track of the action number in
>  a separate predicate.

The order is important, at least to the extent that we need to know
what the *last* action will be -- the one to re-establish
"processing-none". From a more general perspective, I think that a
total ordering could be beneficial for planners because it gives less
options to the search -- it is more explicit about the fact that the
only choice we have is how to instantiate the remaining parameters.

> However, one thing that may be inconvenient about this encoding is
> that only one action can be executed at a time. This would make it
> impossible e.g. for Graphplan to parallelize the execution of
> different actions, because the components of splits for different
> operators have mutex preconditions. Is this going to be a problem
> for FF or other modern planners?

The only potential problem I see is that the heuristic functions may
get "confused" by this construction, ie yield worse search guidance
than on the original planning task. ... we'll just have to see about
this empirically.
}
