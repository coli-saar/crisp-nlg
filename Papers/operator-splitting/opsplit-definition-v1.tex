

\subsection{Definition Version 1}
\label{definition-v1}

\joerg{this entire section is outdated and kept only as a point of
  reference.}



Given an operator $o$ with parameters $P(o)$, define as the
\emph{parameter-groups} the set $G(o) = \{G_1, \dots, G_n\}$ of
parameter subsets $G_i \subseteq P(o)$ that arises from starting with
$G(o) = \emptyset$, then looping over all predicate $p$ in $o$'s
description and inserting the set of $p$'s parameters as a new group
$G_i$.


A \emph{$k$-split} of $o$ is a set $S(o) = \{S_1, \dots, S_k\}$ of
parameter subsets so that: (1) $\bigcup_{j=1}^k S_j = P(o)$; (2) for
all $1 \leq i \leq n$ there exists $1 \leq j \leq k$ so that $G_i
\subseteq S_j$. The \emph{arity} of a split $S(o)$ is $max_{1\leq j
  \leq k} |S_j|$. A split is \emph{arity-minimizing} if there exists
no other split (for any $k$) that has smaller arity. The
\emph{overlap} of a split $S(o)$ is $|\bigcap_{j=1}^k S_j|$. A split
is \emph{overlap-minimizing} if there exists no other split that has
smaller overlap.
%$S'(o)$ so that $max_{1\leq j \leq k} |S_j| > max_{1\leq j \leq k'}
%  |S'_j|$. A $k$-split $S(o)$ is \emph{overlap-minimizing} if there
%  exists no other split (for any $k$) $S'(o)$ so that
%  $|\bigcap_{j=1}^k S_j| > |\bigcap_{j=1}^{k'} S'_j|$.
\joerg{I'm not sure if overlap-minimzation is meaningful, but
  I thought I can just as well state it. I haven't yet thought at all
  about any possible implications between the two, neither whether
  finding minimal splits is hard. I would guess it's not.}


We observe that, given a split, we can re-write $o$ into $k$ operators
$o_1, \dots, o_k$ so that executing $o$ is equivalent to executing
$o_1, \dots, o_k$ in sequence -- note that this is basically inverting
all the known work on macro-operators... :-)


We introduce new $0$-ary predicates $OK$, $do-o_2, \dots,
do-o_k$. $OK$ becomes a precondition of every operator other than $o$,
as well as of $o_1$; it becomes a delete of $o_1$. $do-o_i$ becomes an
add effect of $o_{i-1}$, a precondition of $o_i$, and a delete effect
of $o_i$. The precondition/add list/delete list of $o_i$ simply
consists of those predicates $p$ of $o$ whose parameters are a subset
of $S_i$ -- more precisely, we put each predicates into exactly one
(arbitrary) $o_i$ for which this is the case. \joerg{Unless I'm
  mistaken, this simple construction does the job -- the modified
  planning task has ``the same''set of plans, modulo replacing $o_1,
  \dots, o_k$ with an according instantiation of $o$ (of course we
  need to prove this formally). Generalizing this to splitting up
  several operators should be straightforward. ... I think there will
  also be an inter-action with predicate-splitting as known in logics:
  smaller predicate arity will yield smaller parameter groups and thus
  splits with smaller arity!}


For example, consider the blocksworld operator ``move(A,B,C)'', whose
precondition is ``on(A,B),clear(A),clear(C)'', whose add is ``on(A,C),
clear(B)'' and whose del is ``on(A,B), clear(C)''. A 2-split is given
by ``A,B'' and ``A,C''. We get the actions ``move1(A,B)'' with pre
``OK, on(A,B), clear(A)'', add ``do-move2, clear(B)'', del ``on(A B),
OK''; and ``move2(A,C)''with pre ``do-move2, clear(C)'', add ``on(A
C), OK'', del ``clear(C), do-move2''.



\joerg{I note that this construction may run planners into trouble
  because, somewhere along the way $o_1, \dots, o_k$, a precondition
  may not be satisfied, ie there may not exists any continuation of
  the operator $o$ -- an instantiation of the remaining parameters --
  so that the op execution can be completed. The planner then finds
  itself in a dead-end. One may think of giving the option to ``roll
  back'' an execution. .. well, maybe this is nonsense because
  branching factor inside $o_1, \dots, o_k$ is zero anyway so
  back-tracking is ``easy''. Well, something to keep in mind.}


