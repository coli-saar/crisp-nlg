\newcommand{\tup}[1]{\langle #1 \rangle}



\subsection{Definition: Carlos}
\label{definition-carlos}

Let start by introducing some basic notation

\begin{definition}
An \emph{action} is a pair $a = \tup{P,E}$ where both $P$ and $E$ are finite sets of 
propositional literals. $P$ is the set of \emph{preconditions} of $a$. $E$ is the set of \emph{effects} of $a$. 


A \emph{planning domain} $D$ is a finite set of actions. 
\fxnote{\tiny If we want, we can define something like a `\emph{restricted planning domain}' as a pair $D=\tup{A,R}$ where $A$ is a finite set of actions, and $R$ is a finite set of 
propositional formulas.  We would use $R$ to restrict planning tasks by requesting consistency with $R$.}

A \emph{planning task} is a tuple $\Pi=\tup{D, P}$
where $D$ is a planning domain and  $P=\tup{I,F}$ 
is an \emph{instance description}, consisting of a finite set of literals $I$
(the \emph{initial state}) and the a finite set of literals $O$ (the \emph{final state}). 

\fxnote{\tiny I don't see that we are using objects anywere, so I abtracted from them and I'm only considering propositional literals.}
\end{definition}


\begin{definition}[Reduction]
A reduction of planning domain $D$ to planning domain $D'$ is a tuple
$\mathbf r=\left<f_1, f_2\right>$ of functions
such that for all planning tasks $\Pi = \left<D, P\right>$:
\begin{enumerate}
    \item $f_1$ and $f_2$ are computable functions,
    \item $\left<D', f_1(P)\right>$ is a planning task,
    \item for each plan $\pi'$ of $\left<D', f_1(P)\right>$, $f_2(\pi')$ is
          a plan for $\Pi$,
\end{enumerate}

We say that a reduction $\mathbf r = \tup{f_1,f_2}$ 
\begin{enumerate}
\item is \emph{polynomial} if both $f_1$ and $f_2$ are 
polynomial-time computable;  

\fxnote{\tiny I suppose that \textit{cost} is a function from plans to natural numbers.}
\item \emph{preserves costs} if for each two plans $\pi'_1$ and $\pi'_2$ of $\left<D', f_1(P)\right>$
          it holds that $\textit{cost}(f_2(\pi'_1)) >
          \textit{cost}(f_2(\pi'_2))$ iff $\textit{cost}(\pi'_1) >
          \textit{cost}(\pi'_2)$;

\item \emph{preserves optimal plans} if for at least one optimal plan $\pi^*$ of $\Pi$ there is a
          plan $\pi'$ of $\left<D', f_1(P)\right>$
          whose size is polynomial in the size of $\pi^*$ and
          $\textit{cost}(\pi^*) = \textit{cost}(f_2(\pi'))$.
\end{enumerate}

\end{definition}

\begin{definition}[Reducibility]
A planning domain $D$ is reducible to planning domain $D'$ (written $D\leq D'$)
if there is a reduction $\mathbf r$ from $D$ to $D'$.
\end{definition}

\begin{definition}[Equivalence of planning domains]
Two planning domains $D$ and $D'$ are equivalent (written $D\equiv D'$) iff
$D\leq D'$ and $D'\leq D$.
\end{definition}

Some comments:

\begin{itemize}

\item If we are interesting just in expressivity (when a planning domain can be reduced 
to another) I don't see why we need to impose time and space constrains. 

\item On the other hand, a polynomial reduction preserves complexity bounds. 

\item In the same way, reductions with stronger properties preserve more properties of the planning domain. 

\end{itemize}




