\documentclass{llncs}
\usepackage{times}
\usepackage{helvet}
\usepackage{courier}
\usepackage{epsfig,graphics,latexsym}
\usepackage{amsmath,amssymb,theorem,enumerate}


%% Farbige Kommentar-Felder
\usepackage[usenames]{color} % Only used in comment commands


\newcommand{\commentout}[1]{}



\definecolor{Blue}{rgb}{0,0.16,0.90}
\definecolor{Red}{rgb}{0.90,0.16,0}
\definecolor{DarkBlue}{rgb}{0,0.08,0.45}
\definecolor{ChangedColor}{rgb}{0.9,0.08,0}
\definecolor{CommentColor}{rgb}{0.2,0.8,0.2}
\definecolor{ToDoColor}{rgb}{0.1,0.2,1}

\definecolor{verylightgray}{rgb}{0.91,0.91,0.91}

% *** Use this definition of the command to show the comments ***
\newcommand{\todo}[1]{\textbf{\color{ToDoColor} TODO: #1}}
\newcommand{\joerg}[1]{\textbf{\color{CommentColor} /* #1  (joerg)*/}}
\newcommand{\alex}[1]{\textbf{\color{CommentColor} /* #1  (alex)*/}}
\newcommand{\carlos}[1]{\textbf{\color{CommentColor} /* #1  (carlos)*/}}




%\newenvironment{proof}{\noindent {\bf Proof:}}%
%{\hfill \rule[0.3ex]{1ex}{1ex} \par \addvspace{\bigskipamount}}

\newenvironment{proofsketch}{\noindent {\bf Proof Sketch:}}%
{\hfill \rule[0.3ex]{1ex}{1ex} \par \addvspace{\bigskipamount}}




\title{When Operators Learned How to Split Up}

\author{Koller, Areces, Hoffmann, \& Friends}


\institute{I'm sure they're working somewhere \dots}






\begin{document}


\maketitle


\begin{abstract}
PDDL is arguably a programming language for planners. While that
language, in the case of ``fully automatic'' planners, is limited to a
description of the mechanics of the domain of interest, it is
well-known that different models of the same domain can yield immense
differences in planner performance. Indeed, to a considerable extent
the domain-designing activities of hitherto IPC organizers consisted
of ``spoon-feeding'' the targeted domains to the planners, in that
they used their inside knowledge of planning algorithms to design the
domains in a way so that existing planners could attack them in a
sensible way. However, what if actual ``end-users'', people that want
to use planners but that haven't spent the last 10 years implementing
them, wish to ``spoon-feed'' their domain to the planners?

To some extent, this can be supported by documentation and best
practices explanations, as well as graphical environments for domain
modeling. However, the holy grail would be to provide
\emph{domain-rewriting} techniques, in a manner similar to what is
known as program re-writing: Given an input model designed by the
end-user, can we automatically modify that model to yield better
performance, while still encoding essentially the same domain?

We herein begin to address this question, focussing on one particular
technical detail that affects the pre-processing (and in certain ways
also the search techniques) of many existing planners: the number of
operator parameters. For example, in the development of the Pipesworld
domain as used at IPC 2004, one of the essential steps was to split up
some of the operators, so that each part would have less parameters
and thus be more feasible for common pre-processing techniques. We
herein provide an automatic method that does just that. We evaluate
the method on IPC benchmarks, as well as on the domain of sentence
generation, where pre-processing, in particular related to the number
of operator parameters, is known to be one of the major obstacles.
\end{abstract}




\setcounter{tocdepth}{4}

\tableofcontents





\section{Operator Splitting}



\subsection{Definition Version 1}
\label{definition-v1}

\joerg{this entire section is outdated and kept only as a point of
  reference.}



Given an operator $o$ with parameters $P(o)$, define as the
\emph{parameter-groups} the set $G(o) = \{G_1, \dots, G_n\}$ of
parameter subsets $G_i \subseteq P(o)$ that arises from starting with
$G(o) = \emptyset$, then looping over all predicate $p$ in $o$'s
description and inserting the set of $p$'s parameters as a new group
$G_i$.


A \emph{$k$-split} of $o$ is a set $S(o) = \{S_1, \dots, S_k\}$ of
parameter subsets so that: (1) $\bigcup_{j=1}^k S_j = P(o)$; (2) for
all $1 \leq i \leq n$ there exists $1 \leq j \leq k$ so that $G_i
\subseteq S_j$. The \emph{arity} of a split $S(o)$ is $max_{1\leq j
  \leq k} |S_j|$. A split is \emph{arity-minimizing} if there exists
no other split (for any $k$) that has smaller arity. The
\emph{overlap} of a split $S(o)$ is $|\bigcap_{j=1}^k S_j|$. A split
is \emph{overlap-minimizing} if there exists no other split that has
smaller overlap.
%$S'(o)$ so that $max_{1\leq j \leq k} |S_j| > max_{1\leq j \leq k'}
%  |S'_j|$. A $k$-split $S(o)$ is \emph{overlap-minimizing} if there
%  exists no other split (for any $k$) $S'(o)$ so that
%  $|\bigcap_{j=1}^k S_j| > |\bigcap_{j=1}^{k'} S'_j|$.
\joerg{I'm not sure if overlap-minimzation is meaningful, but
  I thought I can just as well state it. I haven't yet thought at all
  about any possible implications between the two, neither whether
  finding minimal splits is hard. I would guess it's not.}


We observe that, given a split, we can re-write $o$ into $k$ operators
$o_1, \dots, o_k$ so that executing $o$ is equivalent to executing
$o_1, \dots, o_k$ in sequence -- note that this is basically inverting
all the known work on macro-operators... :-)


We introduce new $0$-ary predicates $OK$, $do-o_2, \dots,
do-o_k$. $OK$ becomes a precondition of every operator other than $o$,
as well as of $o_1$; it becomes a delete of $o_1$. $do-o_i$ becomes an
add effect of $o_{i-1}$, a precondition of $o_i$, and a delete effect
of $o_i$. The precondition/add list/delete list of $o_i$ simply
consists of those predicates $p$ of $o$ whose parameters are a subset
of $S_i$ -- more precisely, we put each predicates into exactly one
(arbitrary) $o_i$ for which this is the case. \joerg{Unless I'm
  mistaken, this simple construction does the job -- the modified
  planning task has ``the same''set of plans, modulo replacing $o_1,
  \dots, o_k$ with an according instantiation of $o$ (of course we
  need to prove this formally). Generalizing this to splitting up
  several operators should be straightforward. ... I think there will
  also be an inter-action with predicate-splitting as known in logics:
  smaller predicate arity will yield smaller parameter groups and thus
  splits with smaller arity!}


For example, consider the blocksworld operator ``move(A,B,C)'', whose
precondition is ``on(A,B),clear(A),clear(C)'', whose add is ``on(A,C),
clear(B)'' and whose del is ``on(A,B), clear(C)''. A 2-split is given
by ``A,B'' and ``A,C''. We get the actions ``move1(A,B)'' with pre
``OK, on(A,B), clear(A)'', add ``do-move2, clear(B)'', del ``on(A B),
OK''; and ``move2(A,C)''with pre ``do-move2, clear(C)'', add ``on(A
C), OK'', del ``clear(C), do-move2''.



\joerg{I note that this construction may run planners into trouble
  because, somewhere along the way $o_1, \dots, o_k$, a precondition
  may not be satisfied, ie there may not exists any continuation of
  the operator $o$ -- an instantiation of the remaining parameters --
  so that the op execution can be completed. The planner then finds
  itself in a dead-end. One may think of giving the option to ``roll
  back'' an execution. .. well, maybe this is nonsense because
  branching factor inside $o_1, \dots, o_k$ is zero anyway so
  back-tracking is ``easy''. Well, something to keep in mind.}






\subsection{Initial Definition}
\label{initial-definition}


\joerg{Note: our email text is contained at the end of the tex file as
  a comment}



Given an operator $o$ with parameters $P(o)$, define as the
\emph{parameter-groups} the set $G(o) = \{G_1, \dots, G_n\}$ of
parameter subsets $G_i \subseteq P(o)$ that arises from starting with
$G(o) = \emptyset$, then looping over all predicate $p$ in $o$'s
description and inserting the set of $p$'s parameters as a new group
$G_i$.


A \emph{$k$-split} of $o$ is a set $S(o) = \{S_1, \dots, S_k\}$ of
parameter subsets so that: (1) $\bigcup_{j=1}^k S_j = P(o)$; (2) for
all $1 \leq i \leq n$ there exists $1 \leq j \leq k$ so that $G_i
\subseteq S_j$. The \emph{arity} of a split $S(o)$ is $max_{1\leq j
  \leq k} |S_j|$. A split is \emph{arity-minimizing} if there exists
no other split (for any $k$) that has smaller arity. The
\emph{overlap} of a split $S(o)$ is $|\bigcap_{j=1}^k S_j|$. A split
is \emph{overlap-minimizing} if there exists no other split that has
smaller overlap.
%$S'(o)$ so that $max_{1\leq j \leq k} |S_j| > max_{1\leq j \leq k'}
%  |S'_j|$. A $k$-split $S(o)$ is \emph{overlap-minimizing} if there
%  exists no other split (for any $k$) $S'(o)$ so that
%  $|\bigcap_{j=1}^k S_j| > |\bigcap_{j=1}^{k'} S'_j|$.
\joerg{I'm not sure if overlap-minimzation is meaningful, but
  I thought I can just as well state it. I haven't yet thought at all
  about any possible implications between the two, neither whether
  finding minimal splits is hard. I would guess it's not.}


We observe that, given a split, we can re-write $o$ into $k$ operators
$o_1, \dots, o_k$ so that executing $o$ is equivalent to executing
$o_1, \dots, o_k$ in sequence -- note that this is basically inverting
all the known work on macro-operators... :-)


We introduce new $0$-ary predicates $processing\mbox{-}none$,
$do\mbox{-}o_2, \dots, do\mbox{-}o_k$. We introduce new $1$-are
predicates $arg_1(.), \dots, arg_l(.)$ where $l = |P(o)|$ is the
number of parameters of $o$, and $x_1, \dots, x_l$ is an arbitrary
ordering of $P(o)$. $processing\mbox{-}none$ becomes a precondition of
every operator other than $o$, as well as of $o_1$; it becomes a
delete of $o_1$, and an add effect of $o_k$. $do\mbox{-}o_i$ becomes
an add effect of $o_{i-1}$, a precondition of $o_i$, and a delete
effect of $o_i$. The precondition/add list/delete list of $o_i$
consist: (a) of those predicates $p$ of $o$ whose parameters are a
subset of $S_i$ -- more precisely, we put each predicates into exactly
one (arbitrary) $o_i$ for which this is the case; (b) for any operator
$op_i$ in which parameter $x_j$ appears for the first time in the
sequence $o_1, \dots, o_k$, $arg_j(x_j)$ is inserted into the add
effect of $o_1$, and into the precondition of any $o_g$, $g > i$,
where $o_g$ also uses parameter $x_j$. \joerg{Unless I'm mistaken,
  this simple construction does the job -- the modified planning task
  has ``the same''set of plans, modulo replacing $o_1, \dots, o_k$
  with an according instantiation of $o$ (of course we need to prove
  this formally). Generalizing this to splitting up several operators
  should be straightforward. ... I think there will also be an
  inter-action with predicate-splitting as known in logics: smaller
  predicate arity will yield smaller parameter groups and thus splits
  with smaller arity!}


For example, consider the blocksworld operator ``move(A,B,C)'', whose
precondition is ``on(A,B),clear(A),clear(C)'', whose add is ``on(A,C),
clear(B)'' and whose del is ``on(A,B), clear(C)''. A 2-split is given
by ``A,B'' and ``A,C''. We get the actions ``move1(A,B)'' with pre
``processing-none, on(A,B), clear(A)'', add ``do-move2, clear(B),
arg1(A), arg2(B)'', del ``on(A B), processing-none''; and
``move2(A,C)''with pre ``do-move2, clear(C), arg1(A)'', add ``on(A C),
processing-none, arg3(C)'', del ``clear(C), do-move2''.



\joerg{I note that this construction may run planners into trouble
  because, somewhere along the way $o_1, \dots, o_k$, a precondition
  may not be satisfied, ie there may not exists any continuation of
  the operator $o$ -- an instantiation of the remaining parameters --
  so that the op execution can be completed. The planner then finds
  itself in a dead-end. One may think of giving the option to ``roll
  back'' an execution. .. well dunno if this would help or be
  detrimental. something to keep in mind.}


















%OLD stuff / meta info



\commentout{

%email feedback by alex and carlos, replies by joerg (1 october 2010)

> The encoding of move(A,B,C) in your draft does not work, for two
>reasons. First, nothing in your encoding enforces that between move1
>and move2, only component actions of "move" can be
>executed. Presumably, putdown2 (created by splitting the putdown
>operator) can also be executed whenever OK is false, even if do-move2
>is true.

Nope -- putdown2 has the precondition "do-putdown2" which will not be fulfilled here.

> The second problem is that nothing in the encoding enforces that A
> is bound to the same value in move1 and move2. One could easily
> "split" move(a1,b,c) into the two actions move1(a1,b) and
> move2(a2,c), and the planner would see nothing wrong with it. This
> means that we need a mechanism for keeping track of the variable
> bindings.

>
> We propose the following variant of your mechanism:
>
> move1(A,B)
>  - pre: processing(none), on(A,B), clear(A)
>  - effects: ~processing(none), processing(move), clear(B), ~on(A,B), arg1(A), arg2(B)
>
> move2(A,C)
>  - pre: processing(move), arg1(A), clear(C)
>  - effects: ~processing(move), processing(none), arg3(C), on(A,C)

Yes, good point. I incorporated this into the draft.

> - At each point in the plan, exactly one instance of processing(X)
>is true. While a split of the action "a" is processed, this is the
>atom processing(a); otherwise it is processing(none).

Yep; the same effect is achieved via my construction, where the "do-*"
predicate is true only for one operator (and one part i of that
operator). I did adopt, in the draft, your name "processing-none"
rather than my nondescript "OK".

Note that I say "operator" not "action". This is old-school talk for
distinguishing between the high-level "operators", using variables, in
the usual PDDL file, and the grounded propositional-logic "actions". I
know, it's not very consistent because in PDDL the denotation actually
is "action"... ;-) ... anyway, let's please stick to this.

> If the order in which the component actions of the split are
>  executed is important, one could keep track of the action number in
>  a separate predicate.

The order is important, at least to the extent that we need to know
what the *last* action will be -- the one to re-establish
"processing-none". From a more general perspective, I think that a
total ordering could be beneficial for planners because it gives less
options to the search -- it is more explicit about the fact that the
only choice we have is how to instantiate the remaining parameters.

> However, one thing that may be inconvenient about this encoding is
> that only one action can be executed at a time. This would make it
> impossible e.g. for Graphplan to parallelize the execution of
> different actions, because the components of splits for different
> operators have mutex preconditions. Is this going to be a problem
> for FF or other modern planners?

The only potential problem I see is that the heuristic functions may
get "confused" by this construction, ie yield worse search guidance
than on the original planning task. ... we'll just have to see about
this empirically.
}





\section{Task/Domain Equivalence}

\newcommand{\tup}[1]{\langle #1 \rangle}



\subsection{Definition: Carlos}
\label{definition-carlos}

Let start by introducing some basic notation

\begin{definition}
An \emph{action} is a pair $a = \tup{P,E}$ where both $P$ and $E$ are finite sets of 
propositional literals. $P$ is the set of \emph{preconditions} of $a$. $E$ is the set of \emph{effects} of $a$. 


A \emph{planning domain} $D$ is a finite set of actions. 
\fxnote{\tiny If we want, we can define something like a `\emph{restricted planning domain}' as a pair $D=\tup{A,R}$ where $A$ is a finite set of actions, and $R$ is a finite set of 
propositional formulas.  We would use $R$ to restrict planning tasks by requesting consistency with $R$.}

A \emph{planning task} is a tuple $\Pi=\tup{D, P}$
where $D$ is a planning domain and  $P=\tup{I,F}$ 
is an \emph{instance description}, consisting of a finite set of literals $I$
(the \emph{initial state}) and the a finite set of literals $O$ (the \emph{final state}). 

\fxnote{\tiny I don't see that we are using objects anywere, so I abtracted from them and I'm only considering propositional literals.}
\end{definition}


\begin{definition}[Reduction]
A reduction of planning domain $D$ to planning domain $D'$ is a tuple
$\mathbf r=\left<f_1, f_2\right>$ of functions
such that for all planning tasks $\Pi = \left<D, P\right>$:
\begin{enumerate}
    \item $f_1$ and $f_2$ are computable functions,
    \item $\left<D', f_1(P)\right>$ is a planning task,
    \item for each plan $\pi'$ of $\left<D', f_1(P)\right>$, $f_2(\pi')$ is
          a plan for $\Pi$,
\end{enumerate}

We say that a reduction $\mathbf r = \tup{f_1,f_2}$ 
\begin{enumerate}
\item is \emph{polynomial} if both $f_1$ and $f_2$ are 
polynomial-time computable;  

\fxnote{\tiny I suppose that \textit{cost} is a function from plans to natural numbers.}
\item \emph{preserves costs} if for each two plans $\pi'_1$ and $\pi'_2$ of $\left<D', f_1(P)\right>$
          it holds that $\textit{cost}(f_2(\pi'_1)) >
          \textit{cost}(f_2(\pi'_2))$ iff $\textit{cost}(\pi'_1) >
          \textit{cost}(\pi'_2)$;

\item \emph{preserves optimal plans} if for at least one optimal plan $\pi^*$ of $\Pi$ there is a
          plan $\pi'$ of $\left<D', f_1(P)\right>$
          whose size is polynomial in the size of $\pi^*$ and
          $\textit{cost}(\pi^*) = \textit{cost}(f_2(\pi'))$.
\end{enumerate}

\end{definition}

\begin{definition}[Reducibility]
A planning domain $D$ is reducible to planning domain $D'$ (written $D\leq D'$)
if there is a reduction $\mathbf r$ from $D$ to $D'$.
\end{definition}

\begin{definition}[Equivalence of planning domains]
Two planning domains $D$ and $D'$ are equivalent (written $D\equiv D'$) iff
$D\leq D'$ and $D'\leq D$.
\end{definition}

Some comments:

\begin{itemize}

\item If we are interesting just in expressivity (when a planning domain can be reduced 
to another) I don't see why we need to impose time and space constrains. 

\item On the other hand, a polynomial reduction preserves complexity bounds. 

\item In the same way, reductions with stronger properties preserve more properties of the planning domain. 

\end{itemize}








\subsection{Definition: Alex}
\label{definition-alex}








\subsection{Definition: Gabi}
\label{definition-gabi}

Altough I started from the compilation schemes, I ended up with a definition
that is much closer to reductions (an idea that Joerg also mentioned in the
Skype call).

For my definition, I use the following terminology:
A (PDDL) planning task is a tuple $\left<D, P\right>$
where $D$ is the domain description and $P=(O,I,\gamma)$ 
is the instance description, consisting of the set of objects $O$, the
initial state $I$ and the goal specification $\gamma$. 

\begin{theorem}[Reduction]
A reduction of planning domain $D$ to planning domain $D'$ is a tuple
$\mathbf f=\left<f_P, f_{\pi'}\right>$ of functions
such that for all planning tasks $\Pi = \left<D, P\right>$:
\begin{enumerate}
    \item $\left<D', f_P(P)\right>$ is a planning task,
    \item for each plan $\pi'$ of $\left<D', f_P(P)\right>$, $f_{\pi'}(\pi')$ is
          a plan for $\Pi$,
    \item $f_P$ and $f_{\pi'}$ are polynomial-time computable,
    \item for each two plans $\pi'_1$ and $\pi'_2$ of $\left<D', f_P(P)\right>$
          it holds that $\textit{cost}(f_{\pi'}(\pi'_1)) >
          \textit{cost}(f_{\pi'}(\pi'_2))$ iff $\textit{cost}(\pi'_1) >
          \textit{cost}(\pi'_2)$, and
    \item for at least one optimal plan $\pi^*$ of $\Pi$ there is a
          plan $\pi'$ of $\left<D', f_P(P)\right>$
          whose size is polynomial in the size of $\pi^*$ and
          $\textit{cost}(\pi^*) = \textit{cost}(f_{\pi'}(\pi'))$.
\end{enumerate}
\end{theorem}

\begin{theorem}[Reducibility]
A planning domain $D$ is reducible to planning domain $D'$ (written $D\leq D'$)
if there is a reduction $\mathbf f$ from $D$ to $D'$.
\end{theorem}

\begin{theorem}[Equivalence of planning domains]
Two planning domains $D$ and $D'$ are equivalent (written $D\equiv D'$) iff
$D\leq D'$ and $D'\leq D$.
\end{theorem}

\noindent Some random remarks:
\begin{itemize}
    \item From a practical perspective we are surely interested into generating
        a more expressive domain (a domain which our original domain is 
        reducible to) plus the reduction. However, from a theoretical point of
        view, if we want to show that two domains are \emph{not} equivalent, we
        want to prove that there is no reduction.
    \item Reductions are composable, i.e.\ if $\left<f_P^1, f_{\pi'}^1\right>$
          is a reduction of $D$ to $D_1$ and $\left<f_P^2, f_{\pi'}^2\right>$
          is a reduction of $D_1$ to $D_2$, then $\left<f_P^2\circ f_P^1, 
          f_{\pi'}^2\circ f_{\pi'}^1\right>$ is a reduction of $D$ to $D_2$.
    \item Condition $2$ ensures (together with the existence requirement in
          condition $5$) that there is a plan for $\Pi$ iff there is a
          plan for $\left<D', f_P(P)\right>$.
    \item Condition $4$ ensures that the domains caputure the same idea
          of (relative) plan quality.
    \item Condition $5$ makes the reduction optimality preserving. In practice
          it is probably most of the time the case (and the easiest to show) that
          this condition holds for all plans $\pi$ of $\Pi$ but the current
          definition is sufficient to ensure that 
          \begin{itemize}
            \item for optimal planning, we can use the reduction to find an
                optimal plan if there is a plan.
            \item for satisficing planning, we can use the reduction to
                find a plan if there is a plan.
          \end{itemize}
          However, if we do not extend condition $5$ to all plans $\pi$ of
          $\Pi$, we are not guaranteed that we can find a plan for each
          cost for which a plan in the original task exists.
    \item I still have to think more about for what the time and space
          restrictions are necessary and whether there must be a 
          restriction on the size of $D'$ (currently I do not see, why).
\end{itemize}






\bibliographystyle{plain}
\bibliography{biblio}


\end{document}
