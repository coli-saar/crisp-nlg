

\subsection{Definition: Gabi}
\label{definition-gabi}

Altough I started from the compilation schemes, I ended up with a definition
that is much closer to reductions (an idea that Joerg also mentioned in the
Skype call).

For my definition, I use the following terminology:
A (PDDL) planning task is a tuple $\left<D, P\right>$
where $D$ is the domain description and $P=(O,I,\gamma)$ 
is the instance description, consisting of the set of objects $O$, the
initial state $I$ and the goal specification $\gamma$. 

\begin{definition}[Reduction]
A reduction of planning domain $D$ to planning domain $D'$ is a tuple
$\mathbf f=\left<f_P, f_{\pi'}\right>$ of functions
such that for all planning tasks $\Pi = \left<D, P\right>$:
\begin{enumerate}
    \item $\left<D', f_P(P)\right>$ is a planning task,
    \item for each plan $\pi'$ of $\left<D', f_P(P)\right>$, $f_{\pi'}(\pi')$ is
          a plan for $\Pi$,
    \item $f_P$ and $f_{\pi'}$ are polynomial-time computable,
    \item for each two plans $\pi'_1$ and $\pi'_2$ of $\left<D', f_P(P)\right>$
          it holds that $\textit{cost}(f_{\pi'}(\pi'_1)) >
          \textit{cost}(f_{\pi'}(\pi'_2))$ iff $\textit{cost}(\pi'_1) >
          \textit{cost}(\pi'_2)$, and
    \item for at least one optimal plan $\pi^*$ of $\Pi$ there is a
          plan $\pi'$ of $\left<D', f_P(P)\right>$
          whose size is polynomial in the size of $\pi^*$ and
          $\textit{cost}(\pi^*) = \textit{cost}(f_{\pi'}(\pi'))$.
\end{enumerate}
\end{definition}

\begin{definition}[Reducibility]
A planning domain $D$ is reducible to planning domain $D'$ (written $D\leq D'$)
if there is a reduction $\mathbf f$ from $D$ to $D'$.
\end{definition}

\begin{definition}[Equivalence of planning domains]
Two planning domains $D$ and $D'$ are equivalent (written $D\equiv D'$) iff
$D\leq D'$ and $D'\leq D$.
\end{definition}

\noindent Some random remarks:
\begin{itemize}
    \item From a practical perspective we are surely interested into generating
        a more expressive domain (a domain which our original domain is 
        reducible to) plus the reduction. However, from a theoretical point of
        view, if we want to show that two domains are \emph{not} equivalent, we
        want to prove that there is no reduction.
    \item Reductions are composable, i.e.\ if $\left<f_P^1, f_{\pi'}^1\right>$
          is a reduction of $D$ to $D_1$ and $\left<f_P^2, f_{\pi'}^2\right>$
          is a reduction of $D_1$ to $D_2$, then $\left<f_P^2\circ f_P^1, 
          f_{\pi'}^2\circ f_{\pi'}^1\right>$ is a reduction of $D$ to $D_2$.
    \item Condition $2$ ensures (together with the existence requirement in
          condition $5$) that there is a plan for $\Pi$ iff there is a
          plan for $\left<D', f_P(P)\right>$.
    \item Condition $4$ ensures that the domains caputure the same idea
          of (relative) plan quality.
    \item Condition $5$ makes the reduction optimality preserving. In practice
          it is probably most of the time the case (and the easiest to show) that
          this condition holds for all plans $\pi$ of $\Pi$ but the current
          definition is sufficient to ensure that 
          \begin{itemize}
            \item for optimal planning, we can use the reduction to find an
                optimal plan if there is a plan.
            \item for satisficing planning, we can use the reduction to
                find a plan if there is a plan.
          \end{itemize}
          However, if we do not extend condition $5$ to all plans $\pi$ of
          $\Pi$, we are not guaranteed that we can find a plan for each
          cost for which a plan in the original task exists.
    \item I still have to think more about for what the time and space
          restrictions are necessary and whether there must be a 
          restriction on the size of $D'$ (currently I do not see, why).
\end{itemize}

