\section{A unified perspective on GRE} \label{sec:related}

\begin{figure}
  \centering
  \begin{tabular}{l|l}
    GRE algorithm & DL variant \\ \hline
    \newcite{Dale1995} & only conjunction \\
    \newcite{deemter01:_gener_refer_expres} & propositional connectives \\
    \newcite{dale91:_gener_refer_expres_invol_relat} & \el \\
    \newcite{Krahmer2003} & \el \\
    \newcite{kelleher06:_increm_gener_of_spatial_refer} & \el \\
    \newcite{gardent02:_gener_minim_defin_descr} & \alc
  \end{tabular}
  \caption{DL variants used by different GRE algorithms.}
  \label{fig:related}
\end{figure}

The view of GRE as a problem of computing DL concepts allows us to
organize existing approaches to GRE with respect to the logical
connectives of DL which they can use in the referring expressions they
generate.  We have summarized this for some approaches in
Fig.~\ref{fig:related}.  We already mentioned some of these in
Section~\ref{sec:gre} already.  In addition, the graphs
\newcite{Krahmer2003} use to represent referring expressions can be
seen as concepts of \el\ that are only satisfied at those points at
which they can be embedded, and the descriptions generated by
\newcite{gardent02:_gener_minim_defin_descr} are all concepts of \alc,
without requiring that the bisimulation class must be singleton (to be
able to refer to sets).

Next to the inventory of DL connectives, we can also compare the
algorithms that different approaches use to compute the concepts.  In
a purely propositional setting, where there are no relations, our \el\
algorithm performs essentially the same computation as the Incremental
Algorithm \cite{Dale1995}, i.e.\ split the domain according to the
positive atoms.  However, in the relational case, our algorithms
operate very differently than any of the ones listed in
Fig.~\ref{fig:related}.  Rather than attempting to generate a RE for a
single individual and then doing the same for their neighbors if
necessary, with all the associated risks of infinite regress that this
entails (and that then have to be avoided by additional heuristics),
we successively refine a decomposition of the domain into smaller
classes, effectively computing REs for all individuals at the same
time.  That is, our algorithm is not a search algorithm in the sense
of \newcite{DBLP:conf/ijcai/BohnetD05}, and at least the \alc\
algorithm can therefore compute concepts of minimal relational depth
in worst-case polynomial time.  Perhaps closest in spirit is the
Krahmer et al.\ graph algorithm, which also computes \el\ concepts by
extending them successively; but this algorithm, too, is focused on
reference to a single individual, and it has NP-complete worst-case
complexity. 







%%% Local Variables: 
%%% mode: latex
%%% TeX-master: "dl-gre-08"
%%% End: 
