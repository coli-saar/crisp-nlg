\section{Related Work} \label{sec:related}

\cite{Dale1995}

\cite{deemter01:_gener_refer_expres}

\cite{dale91:_gener_refer_expres_invol_relat} -- avoid infinite
regress by requiring that each semantic atom can be used only once in
the description, i.e.\ no \emph{edge} may be used twice.

\cite{gardent02:_gener_minim_defin_descr}

\cite{Krahmer2003} -- avoid infinite regress automatically because
it doesn't count as progress to add an edge to the subgraph that was
already in it, i.e.\ cleaner version of Dale and Haddock heuristic.
Subgraphs can be seen as DL concepts made up of conjunction and
existential quantifiers (no negation).

\cite{kelleher06:_increm_gener_of_spatial_refer} -- generate GREs with
spatial relations. Polynomial algorithm (check this). Generates REs
for landmarks recursively, i.e.\ suffer from Dale and Haddock infinite
regress in principle. Avoid this by using each individual only once in
the entire description, i.e.\ no \emph{node} may be used
twice. Advantage over ours: take salience of landmarks into account.

%%% Local Variables: 
%%% mode: latex
%%% TeX-master: "dl-gre-08"
%%% End: 
