\section{A unified perspective on GRE} \label{sec:related}

\begin{figure}
  \centering
  \begin{small}
  \begin{tabular}{l|p{0.2\textwidth}}
    GRE algorithm & DL variant \\ \hline
    \newcite{Dale1995} & $\mathcal{CL}$ \\
    \newcite{deemter01:_gener_refer_expres} & $\mathcal{PL}$ \\
    \newcite{dale91:_gener_refer_expres_invol_relat} & \el \\
    \newcite{kelleher06:_increm_gener_of_spatial_refer} & \el \\
    \newcite{gardent02:_gener_minim_defin_descr} & $\mathcal{ELU}_{(\neg)}$\\
%    \newcite{Krahmer2003} & \el\ + equality between individuals
  \end{tabular}
  \end{small}
  \caption{DL variants used by different GRE algorithms.}
  \label{fig:related}
\end{figure}

Viewing GRE as a problem of generating DL concepts offers a new,
unified perspective of the GRE problem: It is the problem of computing
a DL concept with a certain (typically singleton) extension.  Many
existing approaches can be subsumed under this view; we have
summarized this for some of them in Fig.~\ref{fig:related}, along with
the DL fragment they use.  We already discussed some of these
approaches in Section~\ref{sec:gre}.  Furthermore, the non-relational
but negative and disjunctive descriptions generated by
\newcite{deemter01:_gener_refer_expres} are simply concepts of
$\mathcal{PL}$; and \newcite{gardent02:_gener_minim_defin_descr}
generalizes this into generating concepts of $\mathcal{ELU}_{(\neg)}$,
i.e., \el\ plus disjunction and atomic negation.
The approach presented here fits well into this landscape, and it
completes the picture by showing how to generate concepts for
arbitrary bisimulation classes in \alc, which combines all connectives
used in any of these previous approaches.

Perhaps closest in spirit to our approach is 
Krahmer et al.\ graph algorithm~\shortcite{Krahmer2003}, which also computes concepts by
extending them successively.  Interestingly, Krahmer et al.\ themselves
discuss the possibility of a logic based approach by considering graph as
models of a logical language and using logical tools.  They explicitly
mentions hybrid logics which can be seen as very expressive description
logics~\cite{arec:hybr05b}.
Their algorithm, though, focuses on
reference to a single individual and has NP-complete worst-case
complexity.

%The graphs  use to represent referring
%expressions go beyond DL in that they can express equality between
%individuals. For example, Krahmer et al. could distinguish a dog that
%bites itself from dogs that bite other dogs; a distinction which
%cannot be made using description logic formulas. Their referring
%expressions can be seen as formulas of a hybrid logic (another related
%and well-studied family of logics \todo{citation?? or do we want to
%mention hybrid logic at all?}), more specifically the logic \el\ +
%equality between individuals.


Where our approach breaks new ground is in the way these concepts
are computed: It successively
refines a decomposition of the domain into subsets.  In this way, it
is reminiscent of the Incremental Algorithm, which in fact can be seen
as the special case of the \el\ algorithm for the non-relational case.
However, unlike
\newcite{dale91:_gener_refer_expres_invol_relat} and its successors,
we do not have to take special precautions to avoid infinite regress. While Dale
and Haddock's algorithm attempts to generate a RE for a single
individual for successive individuals in the model, our algorithms
consider all individuals in parallel and are guaranteed to always terminate.








%%% Local Variables: 
%%% mode: latex
%%% TeX-master: "dl-gre-08"
%%% End: 
