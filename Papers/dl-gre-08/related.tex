\section{Related Work} \label{sec:related}


\todo{Note: This is \emph{not} search (compare
\cite{DBLP:conf/ijcai/BohnetD05}). }


\begin{figure}
  \centering
  \begin{tabular}{l|l}
    GRE algorithm & DL variant \\ \hline
    \newcite{Dale1995} & conjunctions of atoms \\
    \newcite{deemter01:_gener_refer_expres} & propositional logic \\
    \newcite{dale91:_gener_refer_expres_invol_relat} & \el \\
    \newcite{Krahmer2003} & \el \\
    \newcite{kelleher06:_increm_gener_of_spatial_refer} & \el \\
    \newcite{gardent02:_gener_minim_defin_descr} & \alc
  \end{tabular}
  \caption{DL variants used by different GRE algorithms.}
  \label{fig:related}
\end{figure}

The view of GRE as a problem of computing DL concepts allows us to
organize existing approaches to GRE with respect to the logical
connectives of DL which they can use in the referring expressions they
generate.  This is summarized for some approaches in
Fig.~\ref{fig:related}.  At the most unexpressive end of the spectrum,
\newcite{Dale1995} only generate referring expressions that correspond
to conjunctions of atomic concepts.
\newcite{deemter01:_gener_refer_expres} adds the other propositional
connectives (negation and disjunction) to this, whereas
\newcite{dale91:_gener_refer_expres_invol_relat} and its successors
\cite{Krahmer2003,kelleher06:_increm_gener_of_spatial_refer} add
existential quantifiers.  The graphs used by \newcite{Krahmer2003} can
also be seen as concepts of \el\ that are only satisfied at the points
at which they can be embedded.
\newcite{gardent02:_gener_minim_defin_descr} combines these strands
into an algorithm which computes arbitrary concepts in \alc. \todo{Is
  this really what Claire does?}

In this paper, we have presented two algorithms for concepts in \el\
and \alc.  These algorithms compute referring expressions for all
individuals in the domain at once, and thus there is no danger of
infinite regress.  Other approaches for \el\ and higher logics
typically have to invest some effort into avoiding infinite regress --
e.g.\ by requiring that no edge may be used twice
\cite{dale91:_gener_refer_expres_invol_relat} or that no node may be
used twice \cite{kelleher06:_increm_gener_of_spatial_refer}.  The
graph-based algorithm by \newcite{Krahmer2003} avoids infinite regress
with similar ease as our algorithms -- adding an edge to a subgraph
twice doesn't change the subgraph --, but this comes at the price of
NP-complete complexity.

Because they compute referring expressions for all individuals in the
domain together, our algorithms will be strongest in static settings,
such as the generation of descriptions for museum exhibits, in which
the individuals and their properties don't change much.  Nevertheless,
they remain fast enough for real-time use in dynamic settings.
Another advantage that recent algorithms for relational REs
\cite{Krahmer2003,kelleher06:_increm_gener_of_spatial_refer} have over
ours is that they can take weights and salience into account.  How
this could be incorporated into our algorithms is an interesting
question for future research.


%%% Local Variables: 
%%% mode: latex
%%% TeX-master: "dl-gre-08"
%%% End: 
