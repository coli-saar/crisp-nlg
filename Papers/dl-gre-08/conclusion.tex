\section{Conclusion} \label{sec:conclusion}

In this paper, we have presented a new perspective on GRE as the
problem of computing concepts of description logic with given
extensions.  We have also shown how such concepts can be computed
efficiently by applying existing algorithms for computing bisimulation
classes.  Our algorithms are able to generate 95\% of the
non-relational and 67\% of the relational REs from
\newcite{viethen06:_algor_for_gener_refer_expres}.  They are also
extremely efficient (140 ms to generate REs for all individuals in a
random model of size 100); to our knowledge, these are by far the
fastest runtimes for relational GRE reported in the literature.  We
will make our implementations available online.

Because they compute referring expressions for all individuals in the
domain at once, our algorithms will perform especially strongly in
static settings, such as the generation of descriptions for museum
exhibits, in which the individuals and their properties don't change
much.  In such a setting, REs could be computed once when the
application starts, and then reused throughout.  Nevertheless, they
remain fast enough for real-time use in dynamic settings.  One
interesting question for future research is how bisimulation
algorithms that can incrementally update the bisimulation classes when
the model changes, such as \todo{cite}, can be adapted to GRE.

One finding in our experiments with the Viethen and Dale dataset is
that there is no single ordering that covers all human-produced
descriptions, which seems to be in contrast to Dale and Reiter's
\shortcite{Dale1995} assumption that there is only one ordering for
each given domain.  In fact, it is not even the case that each speaker
consistently uses just one ordering.  An interesting open research
question is thus what factors determine which ordering is used.
Unfortunately, both in the Viethen and Dale dataset and in the TUNA
corpus \cite{deemter06:_build_seman_trans_corpus_for}, only a minority
of referring expressions is relational, maybe because these domains
lend themselves very well to row/column style propositional REs.  We
are currently preparing an experiment to collect REs in a domain in
which relational REs will be more frequent.

\todo{referring to sets}


%%% Local Variables: 
%%% mode: latex
%%% TeX-master: "dl-gre-08"
%%% End: 
