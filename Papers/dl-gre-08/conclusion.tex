\section{Conclusion} \label{sec:conclusion}

referring to sets (or is this trivial?)

where do we talk about mutually constraining descriptions like ``the
rabbit in the hat''?

will make implementation available

first time to actually show efficient runtimes for generation of
relational REs


Because they compute referring expressions for all individuals in the
domain together, our algorithms will be strongest in static settings,
such as the generation of descriptions for museum exhibits, in which
the individuals and their properties don't change much.  Nevertheless,
they remain fast enough for real-time use in dynamic settings.
Another advantage that recent algorithms for relational REs
\cite{Krahmer2003,kelleher06:_increm_gener_of_spatial_refer} have over
ours is that they can take weights and salience into account.  How
this could be incorporated into our algorithms is an interesting
question for future research.


relational descriptions in data are comparatively rare and
inconsistent; not sufficiently annotated in TUNA
\cite{deemter06:_build_seman_trans_corpus_for}; need more. 


Things that we care about but that are never said explicitly so far:

\begin{itemize}
\item in a static scenario, we only have to compute REs once -- and in
  any application that uses many REs before the context changes (e.g.\
  museums), this is exactly what we want to do
\item it is difficult to control the search for concepts by linguistic
  constrtaints
\item our GRE algorithms can take orderings over the propositional and
  relational symbols into account
\end{itemize}


One finding in our experiments with the Viethen and Dale dataset is
that there is no single ordering that covers all human-produced
description, which seems to be in contrast to Dale and Reiter's
\shortcite{Dale1995} assumption that there is only one ordering for
each given domain.  In fact, it is not even the case that each speaker
consistently uses just one ordering.  An interesting open research
question is thus what factors determine which ordering is used.
Unfortunately, both in the Viethen and Dale dataset and in the TUNA
corpus \cite{deemter06:_build_seman_trans_corpus_for}, only a minority
of referring expressions is relational, maybe because these domains
lend themselves very well to row/column style propositional REs.  We
are currently preparing an experiment to collect REs in a domain in
which relational REs will be more frequent.

%%% Local Variables: 
%%% mode: latex
%%% TeX-master: "dl-gre-08"
%%% End: 
