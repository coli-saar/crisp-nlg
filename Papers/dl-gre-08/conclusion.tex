\section{Conclusion} \label{sec:conclusion}

In this paper, we have explored the idea of viewing the generation of
singular REs as the problem of computing a DL formula with a given
extension.  We have shown how such formulas can be computed
efficiently (for \alc\ and \el) by adapting existing algorithms for
computing similarity sets.  The \el\ algorithm is able to generate
95\% of the non-relational and 67\% of the relational REs from
\newcite{viethen06:_algor_for_gener_refer_expres}.  Both algorithms
are extremely efficient (350 ms and 140 ms respectively to generate
relational REs for all individuals in a random model with 100
individuals); to our knowledge, these are by far the fastest runtimes
for relational GRE reported in the literature.  We will make our
implementation available online.

Because they compute referring expressions for all individuals in the
domain at once, our algorithms will perform especially strongly in
static settings, such as the generation of descriptions for museum
exhibits, in which the individuals and their properties don't change
much.  However, even in more dynamic settings, our algorithms have a
chance to outperform search algorithms like Dale and Haddock's in the
average case because they can't get stuck in unproductive branches of
the search space. Nevertheless, one interesting question for future
research is how to incrementally update simulation classes when the
model changes. Similarly, it would be interesting to explore how
linguistic constraints and different attribute orderings can be taken
into account without slowing the algorithms down, and how the
algorithms can be adapted to compute REs for sets. In exploring these
extensions we will be able to draw on a rich body of literature that
considers many variants of simulation algorithms, including for
example incremental simulation algorithms or algorithms for
identifying sets of individuals.


%Nevertheless, one interesting question for future
%research is how to adapt simulation algorithms that can
%incrementally update the simulation classes when the model changes.
%Similarly, it would be interesting to explore how linguistic
%constraints and different attribute orderings can be taken into
%account without slowing the algorithms down, and how the algorithms
%can be adapted to compute REs for sets. 







In experimenting with the Viethen and Dale data, we found that
there is no single ordering that covers all human-produced
descriptions, which seems to be in contrast to Dale and Reiter's
\shortcite{Dale1995} assumption that there is only one ordering for
each given domain.  In fact, it is not even the case that each speaker
consistently uses just one ordering.  An interesting open research
question is thus what factors determine which ordering is used.
Unfortunately, both in the Viethen and Dale dataset and in the TUNA
corpus \cite{deemter06:_build_seman_trans_corpus_for}, only a minority
of referring expressions is relational, maybe because these domains
lend themselves very well to row/column style propositional REs.  We
are currently preparing an experiment to collect REs in a domain in
which relational REs will be more frequent.



%%% Local Variables: 
%%% mode: latex
%%% TeX-master: "dl-gre-08"
%%% End: 
