\section{Introduction} \label{sec:introduction}

The generation of referring expressions (GRE) is one of the most
active and successful research areas in natural language generation.
Starting with Dale and Reiter's incremental algorithm \cite{Dale1995},
various researchers have added extensions such as reference to sets
\cite{Stone2000}, more expressive logical connectives
\cite{deemter01:_gener_refer_expres}, and relational expressions
\cite{dale91:_gener_refer_expres_invol_relat}.

Referring expressions involving relations, in particular, have
received increasing attention recently, particularly in the context of
spatial referring expressions in situated generation (e.g.\
\cite{kelleher06:_increm_gener_of_spatial_refer}), where it seems
particularly natural to use expressions such as ``the book on the
table''.  However, the classical algorithm by
\newcite{dale91:_gener_refer_expres_invol_relat} was recently shown to
be unable to generate satisfying REs in practice
\cite{viethen06:_algor_for_gener_refer_expres}.  Furthermore, the Dale
\& Haddock algorithm and most of its successors
\cite{Krahmer2003,kelleher06:_increm_gener_of_spatial_refer} are
vulnerable to the problem of ``infinite regress'', where the algorithm
jumps back and forth between generating descriptions for two related
individuals infinitely, as in ``the book on the table which supports a
book on the table \ldots''.

In this paper, we propose a fundamentally new way of looking at the
problem of generating referring expressions: We take a referring
expression describing an individual $i$ to be a formula of some
description logic (DL) which is satisfied by $i$ and no other
individual in the universe.  This allows us to organize existing
approaches to GRE in an expressiveness hierarchy.  For instance, the
classical Dale \& Reiter algorithms compute concepts in a description
logic that consists only of conjunctions of atoms, whereas Dale \&
Haddock and other relational algorithms compute concepts in the logic
\el, which permits conjunction and existential quantification.

Furthermore, we show how to addapt algorithms for computing the
\emph{bisimulation classes} of a DL model to the GRE problem.  Intuitively,
bisimulation classes are sets of individuals that satisfy exactly the
same concepts; if an individual is alone in its bisimulation class,
then the concept corresponding to this class represents a uniquely
referring expression.  We present bisimulation classes algorithms for
the description logics \el\ (only conjunction and existential
quantification) and \alc\ (full modal logic).  By essentially
computing referring expressions for all individuals in the domain at
once, these algorithms systematically avoid the infinite regress
problem.  The \el\ algorithm, which generates the same kind of
descriptions as Dale \& Haddock, is capable of generating 67\% of the
relational REs in the
\newcite{viethen06:_algor_for_gener_refer_expres} dataset correctly,
and takes about 400 ms to compute relational REs for all 100
individuals in a random scene.

The paper is structured as follows.  In Section~\ref{sec:bisim}, we
will first define description logic and fundamental concepts such as
bisimulation.  We will then show how to compute bisimulation classes
 and how to apply these algorithms to GRE in Section~\ref{sec:gre}.  In
Section~\ref{sec:discussion}, we evaluate our algorithms and discuss
our results.  Section~\ref{sec:related} compares our approach to
related research; in particular, it shows how various prominent GRE
algorithms can be seen as special cases of ours.
Section~\ref{sec:conclusion} concludes and points to future work.






%%% Local Variables: 
%%% mode: latex
%%% TeX-master: "dl-gre-08"
%%% End: 
