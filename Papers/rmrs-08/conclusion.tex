\section{Conclusion}
\label{sec:conclusion}

In this paper, we motivated and defined \rmrs---a semantic framework
that has been used to represent, compare, and combine semantic
information computed from deep and shallow parsers.  \rmrs\ is
designed to be maximally flexible on the type of semantic information
that can be left underspecified, so that the semantic output of a
shallow parser needn't over-determine or under-determine the semantics
that can be extracted from the shallow syntactic analysis.  Our key
contribution was to lay the formal foundations for a formalism that is
emerging as a standard in robust semantic processing.

Although we have not directly provided new tools for modelling or
processing language, we believe that a cleanly defined model theory
for \rmrs\ is a crucial prerequisite for the future development of
such tools; this strategy was highly successful for dominance
constraints \cite{Althaus_etal:JoA}. We hope that future research will
build upon this paper to develop efficient algorithms and
implementations for solving \rmrs s, performing inferences that enrich
\rmrs s from shallow analyses with deeper information, and checking
consistency of \rmrs s that were obtained from different parsers.

\paragraph{Acknowledgments.} We thank Ann Copestake, Dan Flickinger,
and Stefan Thater for extremely fruitful discussions and the reviewers
for their comments. The work of Alexander Koller was funded by a DFG
Research Fellowship and the Cluster of Excellence ``Multimodal
Computing and Interaction''.

%%% Local Variables: 
%%% mode: latex
%%% TeX-master: "rmrs-08"
%%% End: 
