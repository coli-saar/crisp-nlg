\section{Robust Minimal Recursion Semantics}
\label{sec:rmrs}

% Define syntax of RMRS; model theory of RMRS; solved forms.  Explain
% the role of the RMRS models (= trees) in the semantic construction
% pipeline, in particular with respect to decoding trees into semantic
% representations of the base language formalism.  Present examples in
% MRS syntax first, then in our RMRS language.  Finally, compare our
% version of RMRS against MRS and dominance constraints.

\subsection{RMRS Syntax}

To formalise the basic ideas from Section~\ref{sec:motivation}, 
we start by defining the syntax of the {\sc rmrs} language.
We adopt a
syntax in the style of {\sc clls} (ref), in that labels, anchors and
holes are all represented as {\em node variables}
$\nvar=\{X,Y,X_1,\ldots\}$---the distinction among them comes
from their position in the formulae of an {\sc rmrs} description (see
Definition~\ref{defn:rmrs-syntax}). 
The {\sc rmrs} vocabulary also includes
base variables $\ovar$, consisting of
individual variables $\{x,x_1,y,\ldots\}$ and event variables
$\{e_1,\ldots\}$ (one can also include supersorts to
these, but we forego doing so here).  Finally, the vocabulary includes
0-ary predicate symbols $\pred=\{P,Q,P_1,\ldots\}$.  

\begin{definition}\label{defn:rmrs-syntax}
An {\sc rmrs} is an underspecified description $\varphi$ consisting of a
finite set of atoms 
of the forms in (\ref{eg:atomic-forms}), such that $\varphi$
satisfies the
following three constraints:
\begin{enumerate}
\item   If $\ep{X}{Y}{P}, \ep{Z}{Y}{Q}\in \varphi$, then $X=Z$ and $P=Q$
  (i.e., each predication in $\varphi$ has a unique anchor);
\item   If $\ARG_S(X,v)\in \varphi$, then $X:Y:P\not\in \varphi$ and
  $ARG_{S'}(Y,X)\not\in \varphi$ for any $Y\in \nvar$ and
  $P\in \pred$ (i.e., the first argument to an ARG-relation must be an
  anchor).
\item   If $\ep{X}{Y}{P}\in \varphi$, then $\ep{Y}{Z}{Q},
  \ep{Z}{X}{Q}\not\in \varphi$ for all 
  $Z\in \nvar$ or $Q\in \pred$ (i.e., an anchor cannot be a label, or
  {\em vice versa}).
\end{enumerate}
\begin{examples}
\item   \label{eg:atomic-forms}
$\begin{array}{rrll}
\varphi &::= &\ep{X}{Y}{P} \\ %& \mbox{labelled elementary predication} \\
&| &\ARG_S(X,v) \\ % & \mbox{ARG relation for base variables} \\
&| &\ARG_S(X,Y) \\ % & \mbox{ARG relation for node variables} \\
&| &X \qeq Y \\ % & \mbox{QEQ} \\
&| &v_1 = v_2 \;|\; v_1 \neq v_2 \\ % & \mbox{base variable
                                % (in)equality} \\
&| &X = Y \;|\; X \neq Y \\ % & \mbox{node variable (in)equality} \\
&| &P \spec Q % & \mbox{SPEC relation between predicate symbols}
\end{array}
$
\end{examples}
\end{definition}

The subscript $S$ in the ARG relations stands for a subset of $\N$ which
is either finite or $\N$ 
itself.  The original {\sc rmrs} syntax $\ARG_i$ is 
an abbreviation of $\ARG_{\{i\}}$ and $\ARG_n$ is an abbreviation of
$\ARG_\N$.  Furthermore, $l:a:P(v)$ in the {\sc rmrs}s from
Section~\ref{sec:motivation} and earlier papers (e.g.,
\cite{copestake:2003}) 
is a notational variant of $\{X:Y:P, ARG_{\{0\}}(v)\}$ in
Definition~\ref{defn:rmrs-syntax}.  

\subsection{Model Theory}

The model theory formalises the relationship between an {\sc rmrs} and
the fully specific, alternative logical forms that it describes, as
expressed in the base language.
As suggested in Section~\ref{sec:motivation},  each model $\tau$
corresponds to a unique formula of the base language, conceived as a
tree (see Figure~\ref{fig:1}).  {\em Satisfaction} must then give rise
to a notion of {\em 
  truth} and {\em entailment} that respectively correspond to an {\sc
  rmrs} $\varphi$ being a partial description of $\tau$, and one {\sc
  rmrs} $\varphi$ being more specific than another $\varphi'$ (in the sense that
$\varphi$ imposes more
constraints than $\varphi'$ on the possible fully-specific logical forms).

Since each model corresponds to a (unique) base language formula, defining
models depends on the base-language's vocabulary $\Sigma$.  More
formally, $\Sigma$ is a sorted signature of
base-language constructors $f$ of sorts $s_1\times\ldots
s_n\rightarrow s$,
where $s,s_1,\ldots,s_n\in {\cal S}$, ${\cal S}$ being the sort
hierarchy for $\Sigma$.  We call $n\geq 0$ the {\em arity} of $f$, and
we write $\Sigma_i$ and $\Sigma_{\geq i}$ for the constructors in $\Sigma$
with arity exactly $i$ and at least $i$ respectively.

The models of {\sc rmrs} are then defined to be finite constructor
trees:
\begin{definition}\label{defn:models}
A {\em finite constructor tree} $\tau$ is
a function $\tau:D
\rightarrow \Sigma$ such that $D$ is a \emph{tree domain} (i.e., a subset of
$\N^*$ which is closed under prefix and left sibling) and the number
of children of each node $u \in D$ is equal to the arity of $\tau(u)$.
We define the \emph{sort} $s(u)$ of each node $u$ in $\tau$ bottom-up
by saying that if $u$ has children $u_1,\ldots,u_n$ and $s(\tau(u)) =
s_1 \times \ldots \times s_n \rightarrow s$, then $s(u) = s$ iff
$s(u_i) \leq s_i$ for all $i$.  $\tau$ is \emph{well-sorted} if
all nodes can be assigned sorts in this way.  In what follows, we only consider
well-sorted finite constructor trees.
\end{definition}

The same tree $\tau$ can
represent formulas from different-style base languages (e.g.,
extensional and modal languages, {\sc drt}, etc) by 
leaving the encoding flexible.  But for reasons 
of space, we don't
explore this here; for simplicity, we
assume a sort hierarchy ${\cal S}$ for $\Sigma$ that is
closely related to the sorts of {\sc rmrs} descriptions
themselves.  So ${\cal S}$ contains the sorts
$h$ (for subformulas), $x$ (for individual variables) and $e$ (for event
variables).
We also
assume that $\Sigma$ contains conjunction constructors
$\wedge_i: h\times\ldots h\rightarrow h$ of arity $i$ for every
$i\in \N$.  We write $D(\tau)$ for the tree domain of a
constructor tree $\tau$, and further define the following relations
between nodes in a finite constructor tree:

\begin{definition}\label{defn:dominance}
$u \qeqdom v$ iff $v = u \cdot 2^k$ for some $k \geq 0$ and all
  symbols $\tau(u), \tau(u \cdot 2), \ldots, \tau(u \cdot 2^{k-1})$
  are quantifiers.

$u \wedgedom v$ iff $u \leq v$ (i.e.\ it is a prefix), and all
  symbols on the path from $u$ to $v$ (not including $v$) are one of
  the $\wedge_i \in \Sigma$.
\end{definition}

The {\em satisfaction} relation between an {\sc rmrs} description $\varphi$
and a finite constructor tree $\tau$ is defined in terms of several
assignment functions.  The need for a 
node variable
assignment function
$\alpha:
\nvar
\rightarrow D(\tau)$ is clear; the
constraints on scope defined by the {\sc rmrs} must be resolvable to
$\tau$'s specific
specific scope information, and for any given $\tau$ there may be more
than one way to do this.
Secondly, to ensure that, for
instance, the predications corresponding to {\em every} and {\em cat}
in the {\sc rmrs}
(\ref{eg:cat-pos}b)  partially describe fully-specific
logical forms that accurately record the semantic dependency between
these lexemes in (\ref{eg:cat}a) as well as partially describing
fully-specific logical forms where they are not semantically
dependent (e.g., as required when the same predications are produced
from the {\sc pos} tagged sentence {\em every dog chased some cat}),
the relationship of satisfaction 
should ensure that the variables from $\ovar$ in an
{\sc rmrs} description can
map in a many-to-one way to variables in $D(\tau)$.  So
satisfaction must also be defined in
terms of 
a (0-ary) variable assignment function
$g:\ovar \rightarrow \Sigma_0$ from {\sc rmrs} variables to 
base-language variables that respects the sorts, i.e.\
$s(g(v)) \leq s(v)$ for all $v \in \ovar$.  Finally, satisfaction must
be defined in terms of a relation $\sigma \subseteq
\pred \times \Sigma_{\geq 1}$ that maps each
{\sc rmrs} predicate symbol to a set of constructors from $\Sigma$.

\begin{definition}\label{defn:satisfaction}
Satisfaction of atoms is defined as follows:
$$
\begin{array}{l}
  \tau, \alpha, g, \sigma \models  \ep{X}{Y}{P}
\mbox{ iff}\\
\qquad \tau(\alpha(Y)) \in \sigma(P) \mbox{ and } \alpha(X) \wedgedom
  \alpha(Y) \\
  \tau, \alpha, g, \sigma \models \ARG_S(X,a)
\mbox{ iff exists  $i \in S$ s.t.}\\
\qquad  \alpha(X) \cdot i \in D(\tau) \mbox{ and }
  \tau(\alpha(X) \cdot 
  i) = g(a) \\
  \tau, \alpha, g, \sigma \models \ARG_S(X,Y)
\mbox{ iff  exists $i \in S$ s.t.}\\
\qquad \alpha(X) \cdot i \in D(\tau), \alpha(X) \cdot
  i = \alpha(Y) \mbox{ and }\\
\qquad
s(\alpha(X) \cdot i) = h \\
  \tau, \alpha, g, \sigma \models X \qeq Y
\mbox{ iff } \alpha(X) \qeqdom \alpha(Y) \\
  \tau, \alpha, g, \sigma \models X ={/}\neq Y
\mbox{ iff } \\
\qquad \alpha(X) ={/}\neq \alpha(Y) \\
  \tau, \alpha, g, \sigma \models v_1 ={/}\neq v_2
\mbox{ iff}\\
\qquad g(v_1) ={/}\neq g(v_2) \\
  \tau, \alpha, g, \sigma \models P \spec Q
\mbox{iff $\sigma(P) \subseteq \sigma(Q)$}
\end{array}
$$
A 4-tuple $\tau,\alpha,g,\sigma$ satisfies an underspecified
representation $\varphi$ (written $\tau,\alpha,g,\sigma \models \varphi$)
iff it satisfies all of its
elements.
\end{definition}
Satisfaction encapsulates information about
semantic scope ambiguities, missing
information about 
semantic dependencies and/or missing information about lexical
subcategorisation and lexical senses.   For $j=\{1,2\}$, suppose that
$\tau_j,\alpha_j,g_j,\sigma\models \varphi$ and let $\vocab(\varphi)$
be the vocabulary that features in $\varphi$.  Then 
$\varphi$ exhibits a semantic scope
ambiguity if there is a $Y,Y'\in \vocab(\varphi)$ such that $\alpha_1(Y) \qeqdom
\alpha_1(Y')$ and $\alpha_2(Y')\qeqdom \alpha_2(Y)$.   It exhibits
missing information about semantic dependencies if $\exists v,v'\in
\ovar$ in $\vocab(\varphi)$ such that $g_1(v)=g_1(v')$ and
$g_2(v)\neq g_2(v')$.  It 
exhibits missing lexical subcategorisation information if $\exists
Y\in \vocab(\varphi)$ such that $\tau_1(\alpha_1(Y))$ is a constructor of a
different type from $\tau_2(\alpha_2(Y))$.  And it exhibits missing
lexical sense information if $\tau_1(\alpha_1(Y))$ and
$\tau_2(\alpha_2(Y))$ are different base-language constructors, but of
the same type.

Truth, validity and entailment can now be defined in terms of
satisfiability 
in the usual way:
\begin{definition}\label{defn:entailment}
Let $\tau$ be a model, $\alpha$ a node variable assignment function,
$g$ a 0-ary variable assignment function, $\sigma$ a mapping from
$\pred$ to the powerset of $\Sigma$, and $\varphi,\varphi'$ {\sc rmrs}
descriptions.  Then:
\begin{description}
\item   [truth:] $\tau\models \varphi$ iff $\exists \alpha,g,\sigma$  such
  that $\tau,\alpha,g,\sigma\models \varphi$
\item   [validity:] $\models \varphi$ iff $\forall \tau$, $\tau\models \varphi$.
\item   [entailment:] $\varphi\models \varphi'$ iff $\forall \tau$, if
  $\tau\models \varphi$ then $\tau\models \varphi'$.
\end{description}
\end{definition}
% NOTE: This definition of entailment isn't right, because it allows
% the assignment $\sigma$ from predicates to constructors to vary
% when testing the satisfiability of $\varphi$ vs. $\varphi'$ with
% respect to $\tau$.  I think that to get the notion of entailment we
% want, $\sigma$ should actually be part of the {\sc rmrs} *model*.
% In other words, an {\sc rmrs} model $M$ is a tuple $\langle \tau,
% \sigma\rangle$, where $\tau$ is a finte constructor tree and
% $\sigma$ is a mapping from $\Pred$ to the power set of $\Sigma$.
% Then truth and entailment are defined correctly.  I believe our
% proofs of things like Theorem~\ref{thm:big-one} actually assumes
% that $\sigma$ is constant when interpreting $\varphi$ vs.\
% $\varphi'$, no?

\subsection{Solved Forms}

Following the framework of dominance constraints, one can define a fragment
of {\sc rmrs} descriptions, known as {\em solved forms}, which are
guaranteed to be satisfiable.  For dominance constraints, solved forms
are sets of constraints whose node variables form a forrest under the
dominance relation $\dom$.  Solved forms for {\sc rmrs} must impose
additional constraints, so as to ensure that all the (underspecified)
ARG relations are satisfiable.  Indeed, we help to ensure this by
substituting assuming that {\sc rmrs} solved forms have any variable
equalities substituted into the ARG relations themselves.

\begin{definition}\label{defn:solved-forms}
  An RMRS $\varphi$ is \emph{in solved form} iff:
  \begin{enumerate}
  \item every variable in $\varphi$ is either a hole, a label or an
    anchor (but not two of 
    these);
%% NOTE: The more constrained definition of an {\sc rmrs} description
%% that we have now makes the set of labels and anchors mutually
%% exclusive.  It does not make labels and holes mutually exclusive;
%% nor anchors and holes mutually exclusive
%% however.  
  \item $\varphi$ doesn't contain equality, inequality, and SPEC
    ($\sqsubseteq$) atoms;
%   \item no downward-looking label occurs on the left-hand side of two
%    different EPs;
  \item if $\ARG_S(Y,i)$ is in $\varphi$, then $|S| = 1$;
  \item for any label $Y$ and index set $S$, there
    are no two atoms $\ARG_S(Y,i)$ and $\ARG_S(Y,i')$ in $\varphi$;
  \item no label occurs on the right-hand side of two
    different $\qeqdom$ atoms.
  \end{enumerate}
\end{definition}

\begin{claim}
  Every RMRS in solved form is satisfiable.
\end{claim}
\begin{proof}
by constructing a model $\tau$ that
contains for each predicate $P\in \varphi$ a member of $\sigma(P)$ with
maximum arity, and that satisfies the ARGs and dominances in $\varphi$.
\end{proof}

One can now define the solved forms of an {\sc rmrs}
description:
\begin{definition}  \label{defn:solved-form-of}
  The \emph{syntactic dominance relation} in an {\sc  rmrs} $\varphi$ is the
  reflexive, transitive closure of the binary relation $$
\begin{array}{r c l}
\{(X,Y) & \mid &  \mbox{$\varphi$ contains $X \qeqdom Y$ or}\\
&& \mbox{$ARG_S(X,Y)$ for some
    $S$}\}
\end{array}
$$  
We write $D(\varphi)$ for the syntactic dominance
  relation of $\varphi$.

  An RMRS $\varphi'$ is \emph{a solved form of} the RMRS $\varphi$
  iff $\varphi'$ is in solved form and there is a substitution $s$
  that maps the  node and base variables of $\varphi$ to the node and
  base variables of $\varphi'$ such that
  \begin{enumerate}
  \item $\varphi'$ contains the EP $\ep{X'}{Y'}{P}$ iff there are
    variables $X,Y$ such that $\ep{X}{Y}{P}$ is in $\varphi$, $X' =
    s(X)$, and $Y' = s(Y)$;
  \item for every atom $\ARG_S(X,i)$ in $\varphi$, there is exactly
    one atom $\ARG_{S'}(X',i')$ in $\varphi$ with $X' = s(X)$, $i' =
    s(i)$, and $S' \subseteq S$;
  \item $D(\varphi') \supseteq s(D(\varphi))$.
  \end{enumerate}
\end{definition}

\begin{claim}
  For every model $\tau$ of some {\sc rmrs} $\varphi$, there is a solved form
  $\varphi'$ of $\varphi$ such that $\tau$ is also a model of $\varphi'$.
\end{claim}
\begin{proof}
  will be by constructing a solved form that respects the equalities,
  ARGs, and dominances in $\tau$.
\end{proof}

\begin{prop}
  Every {\sc rmrs} $\varphi$ has only a finite number of solved forms, up to
  renaming of variables.
\end{prop}
\begin{proof}
  Up to renaming of variables, there is only a finite number of
  substitutions on the node and base variables of $\varphi$.  Let $s$
  be such a substitution.  This fixes the set of EPs of any solved
  form of $\varphi$ that is based on $s$ uniquely.  There is only a
  finite set of choices for the subsets $S'$ in condition 2 of
  Def.~\ref{defn:solved-form-of}, and there is only a finite set of
  choices of new $\qeqdom$ atoms that satisfy condition 3.  Therefore,
  the set of solved forms of $\varphi$ is finite.
\end{proof}




%%% Local Variables: 
%%% mode: latex
%%% TeX-master: "rmrs-08"
%%% End: 
