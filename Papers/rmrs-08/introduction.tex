\section{Introduction} \label{sec:intro}

Representing semantics as a logical form that supports automated
inference and model construction is vital for deeper language
engineering tasks, such as dialogue systems.  Logical forms
can be obtained from hand-crafted deep grammars
\cite{butt:etal:1999,copestake:flickinger:2000}, but this
tends to lack robustness: they fail to cover all
words and linguistic constructions and by design do not handle
ill-formed phrases. 
There has thus been a trend recently towards systems for
robust wide-coverage semantic construction (e.g., 
\cite{bos:etal:2004,zettlemoyer:collins:2007}).     
% COMMENT: this issue about optional arguments is really too cryptic.
% Basically, there is no way for assessing by *any* learned language
% processor whether a single word form that appears in
% a number of different syntactic valencies is an example of lexical
% sense ambiguity, or they are all examples of the same sense being
% used in syntactically different ways (i.e., we have optional
% arguments).  Only hand-crafted grammars can tell you this.  So, for
% example, Bos et al (2004) would get bet(x,y) for "I bet 5 pounds",
% and bet(x,y,z) for "I bet you 5 pounds" and bet(x,p) for "I bet it's
% going to rain": I.e., the vocab of the semantic formalism is
% constructed automatically from NL word forms, and the predicate
% symbols in that automatically constructed vocab vary in (a) valency,
% and (b) the semantic dependencies represented by a given argument
% position (e.g., the second argument of "bet" in "I bet 5 pounds" is
% 5 pounds, and for "I bet it's raining" it's "it's raining".
% 

But there are certain semantic phenomena that these robust
approaches don't capture reliably, including quantifier scope,
optional arguments, and long-distance dependencies (for instance,
\newcite{clark:etal:2004} report that the parser used by
\newcite{bos:etal:2004} yields 63\% accuracy on object
extraction; e.g., {\em the man that I met\dots}).  Forcing a robust
parser to make a decision about these phenomena can therefore be
error-prone.  Depending on the application, it may be preferable to
give the parser the option to leave a semantic decision open when it's not
sufficiently informed---i.e., to compute a partial semantic
representation and to complete it later, using information
extraneous to the parser.

In this paper, we focus on an approach to representing partial
semantic information that supports this strategy: Robust Minimal
Recursion Semantics (\rmrs, \cite{copestake:2007a,copestake:2007b}).
\rmrs\ is designed to support underspecification of lexical ambiguity,
scope, and predicate-argument structure.  It is an emerging standard
for representing partial semantic information, and has been applied in
several implemented systems.  For instance, \newcite{copestake:2003}
and \newcite{frank:2004} use it to specify semantic components to
shallow parsers ranging in depth from {\sc pos} taggers to chunk
parsers and intermediate parsers such as {\sc rasp}
\cite{briscoe:etal:2006}.  {\sc mrs} analyses
\cite{copestake:etal:2005} derived from deep grammars, such as the
English Resource Grammar ({\sc erg}, \cite{copestake:flickinger:2000})
are special cases of {\sc rmrs}.

The key contribution we make is to cast {\sc rmrs}, for the first
time, as a logic with a well-defined model theory.  Previously, no
such model theory existed, and so \rmrs\ had to be used in a somewhat
ad-hoc manner which left open what exactly any given \rmrs\
representation actually \emph{means}.  This has hindered quite
practical progress, both in terms of understanding the relationship of
\rmrs\ to other frameworks such as \mrs\ and predicate logic and in
terms of the development of efficient algorithms.  As one application
of the formal concepts that arise from our definitions, we use
entailment to propose a
novel way of characterising consistency of \rmrs\ analyses across
different parsers.

Section~\ref{sec:motivation} introduces \rmrs\ informally and
illustrates why it is necessary and useful for representing semantic
information across deep and shallow language processors.
Section~\ref{sec:rmrs} defines the syntax and model-theory of \rmrs.
We finish in Section~\ref{sec:entailment} by pointing out some avenues
for future research, and conclude in Section~\ref{sec:conclusion}.


\hidden{
The language in which these partial representations are expressed 


Moreover, in a hybrid system that attempts to combine the outputs of
deep and shallow parsers, 



Robust language processors that produce a single conventional logical
form for a given natural language string are beginning to emerge
(e.g.,
\cite{bos:etal:2004,rupp:etal:2000,wong:mooney:2006,zettlemoyer:collins:2005}).
But the output of these systems don't relate to any gold standard deep
parse as produced by expert grammar developers (for instance, while
the training corpus used in \cite{zettlemoyer:collins:2005} features
control phenomena in the language strings, their logical forms don't
represent it).  This makes it hard to judge the logical forms that the
models derive from a linguistic perspective; nor can one integrate
their output with that of a hand-crafted grammar when desired.




This paper focuses on a particular approach to producing partial
semantic information from robust parsers, exemplified in
\cite{copestake:2003,frank:2004}, among others.  Their strategy is to
utilise semantic underspecification to semi-automatically build
semantic components to shallow parsers, so that the output neither
over-determines nor under-determines the semantic information that is
revealed by the (shallow) syntactic analysis.  The semantic formalism
used to express this is Robust Minimal Recursion Semantics

; this is a generalisation of 

 that is designed to be maximally flexible
in the type of semantic information that can be left underspecified:
it can express partial information about semantic scope, the values of
arguments to predicate symbols and/or their argument position, the
arity of the predicate symbols and the sorts of arguments they take.
We show in Section~\ref{sec:motivation} that all these features are
needed when information about lexical subcategorisation or syntactic
dependencies is missing---a characteristic feature of shallow parsers.
Several researchers have demonstrated that {\sc rmrs} is a suitable
framework on which to semi-automatically construct semantic components
to shallow parsers, ranging in depth .






A major motivation for adopting {\sc rmrs} over other techniques for
robustly deriving logical forms is the promise that it can form the
basis for integrating the output of several parsers, and be compared
in particular with the output of a hand-crafted grammar.  This paper
demonstrates the feasibility of this integration for the first time,
by introducing a model theory for {\sc rmrs}, that in turn defines
entailment among {\sc rmrs} representations.  This entailment relation
is also characterised syntactically as an extension of solved forms
(ref to solved forms).  We show that the proof theory and model theory
of {\sc rmrs} that results provides a formal basis for integrating the
semantic output of several shallow parsers, for checking the
satisfiability of a shallow parse, and for testing its compatibility
with a deep parse.


}

%%% Local Variables: 
%%% mode: latex
%%% TeX-master: "rmrs-08"
%%% End: 
