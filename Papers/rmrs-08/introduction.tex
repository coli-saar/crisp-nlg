\section{Introduction} \label{sec:intro}

% What we'll do here.  Problem: RMRS is designed to support building
% semantic components to shallow parsers, and is intended to offer a way
% of integrating semantic representations
% computed by deep and shallow parsing.  We demonstrate the feasibility
% of this for the first time, by introducing a model theory for RMRS,
% defining entailment among RMRS representations and characterising this
% entailment relation 
% syntactically as extension of solved forms (ref to solved forms).  
% We show that the ``compatability'' of
% deep and shallow semantic representations is definable in terms of the
% entailment 
% relation between the RMRS
% representations.

% just so we have a citation here: \cite{Copestake_etal:MRS}

Representing semantics as a logical form that supports automated
inference and model construction is vital for deeper language
engineering tasks, such as dialogue systems.
These need access to the vast array of
semantic information that is captured in conventional logical forms,
including (but not limited to) information about semantic scope and
predicate argument structure.  Hand-crafted grammars yield detailed
logical forms (e.g., \cite{butt:etal:1999,copestake:flickinger:2000}), but deep parsing tends to lack
robustness: hand-crafted grammars fail to cover all words and
linguistic constructions; and unedited text and speech contains
ill-formed phrases, which by design deep grammars do not handle.  
\hidden{
Deep
parsers on their own also lack ambiguity resolution techniques,
producing all possible analyses.  Integration of statistical parse
selection with deep parsing is promising (e.g.,
\cite{toutanova:etal:2004}), but is not a solved problem.
}

Robost language processors that produce a single conventional logical
form for a given natural language string are beginning to emerge
(e.g.,
\cite{bos:etal:2004,rupp:etal:2000,wong:mooney:2006,zettlemoyer:collins:2005}).
But the output of these systems don't relate to any gold standard deep
parse as produced by expert grammar developers (for instance, while
the training corpus used in \cite{zettlemoyer:collins:2005} features
control phenomena in the language strings, their logical forms don't
represent it).  This makes it hard to judge the logical forms that the
models derive from a linguistic perspective; nor can one integrate
their output with that of a hand-crafted grammar when desired.

This paper focusses on a particular approach to producing semantic
information from robust parsers, exemplified in
\cite{copestake:2003,frank:2004}, among others.  Their strategy is to
utilise semantic underspecification to semi-automatically build
semantic components to shallow parsers, so that the output neither
over-determines nor under-determines the semantic information that is
revealed by the (shallow) syntactic analysis.  The semantic formalism
used to express this is Robust Minimal Recursion Semantics ({\sc
  rmrs}, \cite{copestake:2003}); this is an extension of {\sc mrs}
\cite{copestake:etal:2005} that is designed to be maximally flexible
in the type of semantic information that can be left underspecified:
it can express partial information about semantic scope, the values of
arguments to predicate symbols and/or their argument position, the
arity of the predicate symbols and the sorts of arguments they take.
We show in Section~\ref{sec:motivation} that all these features are
needed when information about lexical subcategorisation or
syntactic dependencies is missing---a characteristic feature of
shallow parsers.  Several researchers have demonstrated that {\sc
  rmrs} is a suitable framework on which to semi-automatically
construct semantic components to shallow parsers, ranging in depth
from {\sc pos} taggers (refs) to chunk parsers (frank ref) and
intermediate parsers ({\sc rasp} ref).

A major motivation for adopting {\sc rmrs} over other techniques for
robustly deriving logical forms is the promise that it can form the
basis for integrating the output of several parsers, and be compared
in particular with the output of a hand-crafted grammar.  This paper
demonstrates the feasibility of this integration for the first time,
by introducing a model theory for {\sc rmrs}, that in turn defines
entailment among {\sc rmrs} representations.  This entailment relation
is also characterised syntactically as an extension of solved forms
(ref to solved forms).  We show that the proof theory and model theory
of {\sc rmrs} that results provides a formal basis for integrating the
semantic output of several shallow parsers, for checking the
satisfiability of a shallow parse, and for testing its compatibility
with a deep parse.

%%% Local Variables: 
%%% mode: latex
%%% TeX-master: "rmrs-08"
%%% End: 
