\documentclass[11pt]{article}

\usepackage{jair, theapa}

\usepackage{times}
\usepackage{url}
\usepackage{graphicx}
\usepackage{bm}
\usepackage{multirow}

\input epsf %% at top of file


\usepackage[algoruled, linesnumbered,noend]{algorithm2e}

\newcommand{\gM}{\ensuremath{\mathcal{M}}}
\newcommand{\gL}{\ensuremath{\mathcal{L}}}

\newcommand{\C}{\ensuremath{\mathcal{C}}\xspace}
\newcommand{\M}{\ensuremath{\mathcal{M}}\xspace}
\newcommand{\RE}{\ensuremath{\mathit{RE}}\xspace}

\newtheorem{definition}{Definition}[section]
\newtheorem{theorem}{Theorem}[section]

\newcommand{\todo}[1]{\textbf{(#1)}}
\newcommand{\ignore}[1]{}

\newcommand{\el}{\ensuremath{\mathcal{EL}}\xspace}
\newcommand{\alc}{\ensuremath{\mathcal{ALC}}\xspace}
\newcommand{\form}{\mathsf{form}\xspace}
\newcommand{\prop}{\ensuremath{\mathsf{prop}}\xspace}
\newcommand{\simm}{\textsf{sim}}
\newcommand{\rel}{\ensuremath{\mathsf{rel}}\xspace}
\newcommand{\propm}{\ensuremath{\mathsf{prop}^\gM\xspace}}

\newcommand{\interp}[1]{|\!|#1|\!|}


\jairheading{0}{0000}{0-00}{0/0}{0/0}
\ShortHeadings{Referring Expressions as Formulas of Description Logic}
{Areces, Koller \& Striegnitz}
\firstpageno{1}

\begin{document}

\title{Referring Expressions as Formulas of Description Logic}

\author{\name Carlos Areces \email areces@loria.fr \\
       \addr INRIA Nancy Grand Est\\
       Nancy, France
       \AND
       \name Alexander Koller \email koller@coli.uni-sb.de \\
       \addr Saarland University\\
       Saarbr\"{u}cken, Germany
       \AND
       \name Kristina Striegnitz \email striegnk@union.edu \\
       \addr Union College\\
       Schenectady, NY, USA}

% For research notes, remove the comment character in the line below.
% \researchnote

\maketitle


\begin{abstract}
In this paper, we propose to reinterpret the problem of generating
referring expressions (GRE) as the problem of computing a formula in a
description logic that is only satisfied by the referent. This view
offers a new unifying perspective under which existing GRE algorithms
can be compared. We also show that by applying existing algorithms for
computing simulation classes in description logic, we can obtain
extremely efficient algorithms for relational referring expressions
without any danger of running into infinite regress.
\end{abstract}



\input introduction
\input bisim
\input gre
\input discussion
\input related
\input conclusion



\acks{We are grateful to Hector Geffner (who
independently suggested to view GRE as computation of DL formulas),
Kees van Deemter, and Emiel Krahmer for interesting discussions.  We
also thank Jette Viethen and Robert Dale for making their corpus
available, and the reviewers for their comments.
}


\appendix
\section*{In case we have an appendix.}

[section ommitted]



\vskip 0.2in
\bibliography{bibliography}
\bibliographystyle{theapa}

\end{document}


