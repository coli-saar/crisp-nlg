
\section{NLG as Planning} \label{sec:nlg-as-planning}

The task of generating natural language from semantic representations
is typically split up into two parts: the \emph{domain planning} task,
which selects the information that is to be conveyed and structures it
into sentence-sized chunks, and the \emph{sentence generation} tasks,
which then translates each of these chunks into natural language
sentences. The sentence generation task is often split up into two
parts of its own---the \emph{sentence planning} task, which enriches
the input by e.g.\ determining how to refer to objects and selects
some lexical material, and the \emph{surface realization} task, which
maps the enriched meaning representation into a sentence using a
grammar. The chain of domain planning, sentence planning, and surface
realization is sometimes called the ``NLG pipeline''
\cite{reiter00building}. 

Viewing generation as a planning problem has a long tradition in the
NLG literature, especially in discourse planning. \todo{Sketch
  Perrault et al here. Say something very brief about Appelt, Hovy,
  Moore. Say why all this didn't really work.}

Most recently, interest in generation as planning has been revived for
NLG tasks on various levels of the NLG pipeline. Below, we will look
in some detail at \cite{KolSto07}'s recasting of sentence generation
as planning; furthermore, \cite{Steedman-Petrick:07} and
\cite{benotti08b} have applied planning to \todo{something}. Notice
that although all of these approaches encode ``communicative actions''
as planning operators, they still solve fundamentally different
problems. For instance, Perrault et al.\ tackled discourse planning
with planning operators that encode speech acts, each of which
represents the meaning of a whole sentence and models the contribution
of this sentence to the intentional structure of the entire
discourse. By contrast, Koller and Stone solve a sentence generation
problem using planning operators that encode utterances of single
words and model the contribution of each word to the grammatical
structure of the sentence. Whether these different levels of ``NLG as
planning'' could be combined into a single planning problem, and
whether this would make computational sense, is an open question.

Problems in NLG at all levels of description can become
computationally complex very easily. For instance, \citet{KolStr02}
showed that even surface realization is NP-complete under reasonable
assumptions. This makes it necessary to investigate methods for
guiding the search, which makes modern planning techniques a promising
approach to NLG. The point of this paper is to investigate whether
existing planners fulfill this promise of efficiency for the
application to NLG.


%%% Local Variables: 
%%% mode: latex
%%% TeX-master: "manuscript"
%%% End: 
