\section{NLG as Planning} \label{sec:nlg-as-planning}

The task of generating natural language from semantic representations
(NLG) is typically split up into two parts: the \emph{discourse
  planning} task, which selects the information that is to be conveyed
and structures it into sentence-sized chunks, and the \emph{sentence
  generation} tasks, which then translates each of these chunks into
natural language sentences. The sentence generation task is often
split up into two parts of its own---the \emph{sentence planning}
task, which enriches the input by e.g.\ determining how to refer to
objects and selects some lexical material, and the \emph{surface
  realization} task, which maps the enriched meaning representation
into a sentence using a grammar. The chain of domain planning,
sentence planning, and surface realization is sometimes called the
``NLG pipeline'' \cite{reiter00building}.

Viewing generation as a planning problem has a long tradition in the
NLG literature.  \citet{perrault80} presented an approach to discourse
planning in which the planning operators represent individual speech
acts such as ``request'' and ``inform''. This idea was later worked
out further, e.g.\ by \citet{young94dpocl}. On the other hand,
researchers such as \citet{appelt:planning} and \citet{hovy88} used
techniques from hierarchical planning to expand a high-level plan
consisting of speech acts into more detailed specifications of
individual sentences. Their systems also covered some aspects of
sentence planning. However, these systems used very expressive logics,
which were designed to reason about beliefs and intentions, to
represent the planning state and the planning operators. Furthermore,
most of them used ad-hoc planning algorithms with rather naive search
strategies, which did not scale to realistic inputs. As a consequence,
the NLG-as-planning approach was mostly marginalized throughout the
1990s.

However, there has recently been a string of publications by various
authors with renewed interest in the generation-as-planning approach,
motivated by the recent development of efficient and expressive
planners. Below (Section~\ref{sec:domain-crisp}) we will look in some
detail at \cite{KolSto07}'s approach to sentence generation (i.e., the
sentence planning and surface realization modules of the pipeline) as
planning. In addition, \citet{Steedman-Petrick:07} revisit the
analysis of indirect speech acts with modern planning technology;
\citet{benotti08b} uses planning to explain the accommodation of
presuppositions; and
\citet{brenner08:_contin_multiag_plann_approac_to_situat_dialog} use
multi-agent planning to model the joint problem-solving behavior of
agents in a situated dialogue. These approaches focus on different
issues than the 1980's NLG-as-planning literature. But the main
difference is that they apply \emph{existing}, well-understood
planning approaches to linguistic problems because these offer new
modeling tools, and because they hope that modern planners can solve
the hard search problems that arise in NLG \citep{KolStr02}
efficiently. The point of this paper is to investigate whether
existing planners fulfill this latter promise.


%%% Local Variables: 
%%% mode: latex
%%% TeX-master: "manuscript"
%%% End: 
