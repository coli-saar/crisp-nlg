\section{Discussion}
\label{sec:discussion}

We can draw both positive and negative conclusions from our
experiments about the state of planning for modern NLG
applications. On the one hand, we found that modern planners are very
good at dealing with the \emph{search} problems that arise in the
NLG-based planning problems we investigated.
In the sentence generation domain, FF's Enforced
Hill-Climbing strategy finds plans corresponding to 25-word sentences
in about a second. It is hard to compare this number to a baseline
because there are no shared benchmark problems, but FF's search
performance is similar to that of a greedy, incomplete special-purpose
algorithm, and competitive to other sentence generators as well. Thus
research on search strategies for planning has paid off; in
particular, the Enforced Hill-Climbing heuristic outperforms the
best-first strategy to which FF 2.3 switches for some problem
instances. Similarly, SGPLAN's performance on the GIVE domain is very
convincing and fast enough for many instances of this application.
% I don't think we want to talk about this yet - AK
%; the search time reported
%by most planners was often a very small fraction of the total running
%time of the planner. 

On the other hand, each of the off-the-shelf planners we tested spent
substantial amounts of time on preprocessing. This is most apparent in the
sentence generation domain, where the planners spent almost their entire
runtime on grounding the predicates and operators for some problem instances.
This effect is much weaker in the GIVE domain, which has a much smaller
number of operators and less interactions between the predicates in the
domain. However, our GIVE experiments also illustrate that altering the
structure of a domain, even minimally, can significantly change a planner's
performance on a problem. For instance, in some of our GIVE experiments with
extra grid positions, increasing the number of buttons in the world, while
keeping the dimensions of the grid fixed, resulted in a significantly larger
search time while the preprocessing time remained essentially unchanged.

While the GIVE domain can be defined in such a way that the number of
operators is minimized, this is not possible for an encoding of a domain in
which the operators model the different communicative actions that the NLG
system can use. For instance, in the sentence generation domain, the XTAG
planning problem for $k=2$ and $n=5$ consists of about 1000 operators for the
different lexicon entries for all the words in the sentence, some of which
take four parameters. It is not unrealistic to assume a knowledge base with a
few hundred individuals. All this adds up to trillions of ground
instances: a set which is completely infeasible to compute naively.

Of course, it would be premature to judge the usefulness of current planners
as a whole, based on just two NLG domains. Nevertheless, we believe that the
structure of our planning problems, which are dominated by large numbers of
operators and individuals, is typical of NLG-related planning problems as a
whole. This strongly suggests that while current planners are able to manage
many of the search problems in the domains we looked at, they are still
unusable for practical NLG applications because of the time they spend on
preprocessing. In other words, the state of generation-as-planning research
is still not in a much better position than it was in the 1980s. 

We are also aware that the time a planner invests in preprocessing can pay
off during search, and that such techniques have been invaluable in improving
the overall running time of modern planners. However, we still suggest that
the inability of current planners to scale to larger domains limits their
usefulness for applications beyond NLG as well. Furthermore, we feel that the
problem of preprocessing receives less research attention than it deserves:
if the problem is scientifically trivial then we challenge the planning
community to develop more efficient implementations that only ground
operators by need; otherwise, we look forward to future publications on this
topic. To support this effort, we offer our planning domains as benchmarks
for future research and competitions.\footnote{The PDDL problem generators
for our NLG domains are available at
\url{http://www.coli.uni-saarland.de/~koller/projects/crisp}.}

Finally, we found it very convenient that the recent International Planning
Competitions provide a useful entry point for selecting and obtaining current
planners. Nevertheless, our experiments exposed several bugs in the planners
we tested, which required us to change their source code to make them scale
to our inputs. We also found that different planners that solve the same
class of planning problems (e.g., STRIPS, ADL, etc.) sometimes differ in the
variants of PDDL that they support. These differences range from fragments of
ADL that can be parsed, to sensitivity to the order of declarations and the
use of ``objects'' rather than ``individuals'' as the keyword for declaring
the universe. We propose that the case for planning as a mature technology
with professional-quality implementations could be made more strongly if such
discrepancies were harmonized.



%%% Local Variables: 
%%% mode: latex
%%% TeX-master: "manuscript"
%%% End: 
