\section{Discussion}
\label{sec:discussion}

There are both positive and negative conclusions about the state of
planning for modern NLG applications that we can draw from these
experiments.

On the one hand, we found that modern planners are very good at
dealing with the \emph{search} problems that are involved in the
NLG-based planning problems we looked at. In the sentence generation
domain, FF's Enforced Hill-Climbing strategy finds plans corresponding
to 25-word sentences in about a second. It is hard to compare this
number to a baseline because there are no shared benchmark problems;
but FF's search outperforms a greedy search algorithm, and the
performance is competitive with other sentence generators, which
similarly employ special-purpose algorithms. Thus research on search
strategies for planning has paid off: In particular, the Enforced
Hill-Climbing heuristic outperforms the best-first strategy to which
FF 2.3 switches for some problem instances. Similarly, SGPLAN's
performance on the GIVE domain is very convincing and fast enough for
the application.

However, each of the off-the-shelf planners we tested spent
substantial amounts of time on preprocessing. This is most apparent in
the sentence generation domain, where the planners spent almost their
entire runtime on grounding the predicates and operators for some
instances. This effect is much weaker in the GIVE domain, which has a
much smaller number of operators; but as we added dummy predicates,
the ratio of time that the planners spent on preprocessing rose in a
similar way. While the GIVE domain can be carefully defined in such a
way that the number of operators is minimized, this is not possible
for a domain in which the operators encode the different communicative
actions that the NLG system can use. For instance, in the sentence
generation domain, the XTAG planning problem for $k=2$ and $n=5$
consists of about 1000 operators for the different lexicon entries for
all the words in the sentence, some of which take four parameters. It
is not unrealistic to assume a knowledge base with a few hundred
individuals. All this adds up to trillions of ground instances, i.e.\
it is completely infeasible to compute ground instances naively.

Of course, it would be premature to judge the usefulness of current
planners for NLG as a whole based on just two NLG
domains. Nevertheless, we believe that the structure of our planning
problems, which are dominated by a large number of operators and
individuals, is typical for NLG-related planning problems as a
whole. This suggests strongly that while current planners deal well
with the search problems we have looked at, they are still unusable
for practical NLG applications because of the time they spend on
preprocessing. In other words, generation-as-planning research is
still not in a much better position than it was in the 1980s. While we
are aware that the time that a planner invests in preprocessing can
pay off during search, we still suggest that the inability of current
planners to scale to larger domains still limits their usefulness for
applications beyond NLG as well. Furthermore, we feel that
preprocessing gets less research attention than it deserves: If it is
scientifically trivial, we challenge the planning community to come up
with implementations that only ground operators by need; otherwise, we
look forward to publications on this topic. To support this, we offer
our planning domains as benchmarks for future competitions. They are
available at \todo{upload them somewhere}

Finally, we found it very convenient that the recent IPC planning
competitions provide an entry point for selecting and obtaining
current planners. Nevertheless, our experiments exposed several bugs
in the planners we tested, and required us to make changes to the
source code to make them scale to our inputs. We also found that
different planners differ in the variants of PDDL that they can
process. These differences range from fragments of ADL that can be
parsed to sensitivity to the order of declarations and the use of
``objects'' rather than ``individuals'' as the keyword for declaring
the universe. We propose that the case for planning as a mature
technology with professional-quality implementations could be made
more strongly if such discrepancies were harmonized.



%%% Local Variables: 
%%% mode: latex
%%% TeX-master: "manuscript"
%%% End: 
