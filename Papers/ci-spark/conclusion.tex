
\section{Conclusion}
\label{sec:conclusion}

In this paper, we introduced two novel planning domains arising from
problems in natural language generation: the sentence generation domain and
the Generating Instructions in Virtual Environments (GIVE) challenge
domain. We also reported the results of a set of experiments designed to
evaluate the suitability of off-the-shelf planners (particularly FF and
SGPLAN) in these domains.

Our results were mixed. While the planners we tested do an impressive job
of controlling the complexity of search in many domains, they also suffer
from practical problems that limit their performance in unexpected ways: in
both of our NLG domains the grounding step performed by FF and SGPLAN
dominates the time it takes to perform the search itself. As future work,
we hope to extend our analysis to other planners that do not perform the
initial grounding step, planners such as SHOP2
\citep{DBLP:journals/jair/NauAIKMWY03} that have traditionally performed
well on a wide range of domains, and new planners introduced in the latest
edition of the IPC.\footnote{See
 \url{http://ipc.informatik.uni-freiburg.de/} for details of the 2008
 International Planning Competition (deterministic track).}

The NLG community's recent interest in planning presents a valuable
opportunity for planning researchers. Domains like GIVE highlight certain
challenges, such as plan execution monitoring and plan presentation (i.e.,
summarisation and elaboration), but also offer a platform on which such
technologies can be evaluated in experiments with human users. Although we
have focused on classical planning problems in this work, research related
to reasoning under uncertainty, resource management, and planning with
knowledge and sensing, can also be investigated in these settings. As such,
we believe our domains would provide suitable challenges for planners
entered in future editions of the IPC. Provided some of the challenges we
have identified can be addressed, NLG-inspired problem domains offer a
constructive platform for the planning community to showcase their
techniques to a wider audience---and to improve the quality of their tools
for real-world planning tasks. 

\todo{planning claims general applicability; it's clear that IPC
  benchmarks focus areas of improvement; but it's not so clear to what
  extent this limits usefulness for non-IPC application areas; this
  paper evaluates this (for the first time??)}



\section*{Acknowledgements}

This work arose in the context of the Planning and Language Interest Group
at the University of Edinburgh. The authors would like to thank all members
of this group, especially Hector Geffner and Mark Steedman, for interesting
discussions. This work was partially supported by the DFG Research
Fellowship ``CRISP: Efficient integrated realization and microplanning''
and by the European Commission through the PACO-PLUS project
(FP6-2004-IST-4-27657).


%%% Local Variables: 
%%% mode: latex
%%% TeX-master: "manuscript"
%%% End: 
