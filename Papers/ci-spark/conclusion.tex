\section{Conclusion}
\label{sec:conclusion}

In this paper, we have investigated the usefulness of current planning
technology to natural language generation, an application area with a
long tradition that has recently experienced renewed interest from NLG
researchers. In particular, we have evaluated the performance of
several off-the-shelf planners on a series of planning domains that
arise in the context of sentence generation and situated instruction
generation.

Our results were mixed. While some of the planners we tested---in
particular, FF and SGPLAN, in different domains---did an impressive
job of controlling the complexity of the search, we also found that
all planners we tested spent too much time on preprocessing to be
useful. In the sentence generation domain, FF spent 90\% of its
runtime on computing the ground instances of the planning operators;
in the instruction-giving domain, which is very similar to Gridworld,
a similar effect happened once we added more predicates. As things
stand, we found that this overly long preprocessing time makes current
planners inapplicable to NLG application for any but the smallest
instances. Users who come to planning from the outside, such as NLG
researchers, take planners as a black box. This means that search
efficiency alone is not helpful when other modules of the planner are
slow. From this perspective, we propose that the planning community
might spend some attention on optimizing the preprocessing module with
similar vigor as the search itself. In particular, we propose to
investigate planning algorithms that do not have to rely on grounding
out all operators before starting the search.

NLG and planning have a long history in common. The recent surge in
NLG-as-planning research presents valuable opportunities for both
sides. Clearly, NLG researchers who apply planning technology will
benefit directly from any improvements in planner effiency. But
conversely, NLG may be a worthwhile application area for planning
researchers to keep in mind. Domains like GIVE highlight certain
challenges, such as plan execution monitoring and plan presentation
(i.e., summarisation and elaboration), but also offer a platform on
which such technologies can be evaluated in experiments with human
users. Further, although we have focused on classical planning
problems in this work, research related to reasoning under
uncertainty, resource management, and planning with knowledge and
sensing, can also be investigated in these settings. As such, we
believe our domains would provide interesting challenges for planners
entered in future editions of the IPC.



\section*{Acknowledgements}

This work arose in the context of the Planning and Language Interest
Group at the University of Edinburgh. The authors would like to thank
all members of this group, especially H{\'{e}}ctor Geffner and Mark
Steedman, for interesting discussions. We also thank our reviewers for
their insightful and challenging comments.  This work was supported by
the DFG Research Fellowship ``CRISP: Efficient integrated realization
and microplanning'', the DFG Cluster of Excellence ``Multimodal
Computing and Interaction'', and by the European Commission through
the PACO-PLUS project (FP6-2004-IST-4-27657).


%%% Local Variables: 
%%% mode: latex
%%% TeX-master: "manuscript"
%%% End: 
