\section{Introduction}
\label{sec:introduction}

Natural language generation (NLG; \citealp{reiter00building}) is one
of the major subfields of natural language processing, concerned with
computing natural language sentences or texts that convey a given
piece of information to the user. This task can be viewed as a problem
of achieving a (communicative) goal by successively applying
(communicative) actions, and thus has clear intuitive parallels to
automated planning.

This view of generation as planning has a long tradition in NLG
\citep{perrault80,appelt:planning,hovy88,young94dpocl}. Despite their
theoretical attractiveness, these early approaches never gained
mainstream use in applied systems, in part because of the inefficiency
of the planners that were available at the time. However, the recent
improvements in planner efficiency and expressiveness have sparked a
renewed interest in applying modern planning techniques to NLG---both
on the level of speech act planning
\citep{Steedman-Petrick:07,benotti08b} and and on the level of
sentence generation \citep{KolSto07}. The goal of this paper is to
determine, by looking at some representative generation problems,
whether planning has advanced to the point that this enterprise can be
successful this time---and thus, to determine the usefulness of
current planning technology to an application area that has a long
tradition, but is not currently in the focus of the planning
competitions.

The focus of this paper is twofold. First, we present two generation
problems that have recently been cast as planning problems: the
sentence generation task and the GIVE task. In the sentence generation
task, the goal is to generate a single sentence that expresses a given
meaning. \citet{KolSto07} cast this task as a planning problem, where
the plan encodes the necessary sentence and the actions correspond to
uttering individual words.  In the GIVE domain (``Generating
Instructions in Virtual Environments''), we describe a new shared task
that was recently posed as a challenge for the NLG community
\citep{ByrKolStrCasDalMooObe09}.  GIVE uses planning as part of an NLG
system that generates natural-language instructions to guide a user in
performing a given task in a virtual environment.

Second, we discuss some of our experiences using off-the-shelf
planners in these two domains. In particular, we explore the
efficiency of FF \citep{HoffmannNebel01} and SGPLAN
\citep{hsu06:_new_featur_in_sgplan_for}: two planners that are readily
available, support an expressive subset of the Planning Domain
Definition Language (PDDL; McDermott et al.~\citeyear{PDDL}) for
encoding domains, and have been successful on the benchmarks and
problems in the International Planning Competition (IPC)\footnote{See
  \url{http://ipc.icaps-conference.org/} for information about past
  editions of the IPC.}. We test these planners on a range of problem
instances in our planning domains, and compare our results against an
ad-hoc Java implementation of GraphPlan \citep{Blum1997}.

Overall, our findings are mixed. On the one hand, we demonstrate that
modern planners can easily handle the \emph{search} problems that
arise in NLG on realistic inputs, which is promising given that the
sentence generation task, at least, is NP-complete
\citep{KolStr02}. On the other hand, the same planners spend
tremendous amounts of time on preprocessing: Because our domains use a
large number of actions and constants, FF spends 90\% of its runtime
in the sentence generation domain on grounding out all the actions and
literals. As a consequence, the off-the-shelf planners we looked at
are too slow to be useful in the NLG application -- but this is due to
implementation issues rather than the quality of the search
algorithms. We offer our NLG domains as challenges for the planning
community in the hope that these implementation issues might gain more
attention in the future.

This paper is structured as follows. We begin by briefly reviewing
some of the literature on NLG as planning
(Section~\ref{sec:nlg-as-planning}). Then we define introduce the
planning problems arising in two particular NLG tasks in more detail
in Section~\ref{sec:domains}. We report on some experiments with these
planning problems in Section~\ref{sec:experiments}, and discuss the
results in Section~\ref{sec:discussion}. Section~\ref{sec:conclusion}
concludes.



%%% Local Variables: 
%%% mode: latex
%%% TeX-master: "manuscript"
%%% End: 
