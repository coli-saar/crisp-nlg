
%%%
%%% Abstract
%%%
\begin{abstract}
  Natural language generation (NLG) is a field of computational
  linguistics that has a long tradition as an application area of
  planning systems. While this connection had not received much
  attention for a while, interest in it has been renewed recently in
  several publications. In this paper, we investigate the extent to
  which these new NLG systems profit from the recent advances in
  planner efficiency. Our findings are mixed: While modern planners do
  very well on solving the search problems that come up in our NLG
  experiments, their overall runtime is dominated by the grounding
  step they perform as preprocessing. Because this makes current
  off-the-shelf planners unusably slow for nontrivial NLG problem
  instances, we offer our domains as challenges for the planning
  community.
\end{abstract}

\bigskip\noindent
\textbf{Keywords:} natural language generation, planning


%%% Local Variables: 
%%% mode: latex
%%% TeX-master: "manuscript"
%%% End: 
