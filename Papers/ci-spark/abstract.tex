\begin{abstract}
  Natural language generation (NLG) is a major subfield of
  computational linguistics with a long tradition as an application
  area of automated planning systems. While things were relatively
  quiet with the planning approach to NLG for a while, several recent
  publications have sparked a renewed interest in this area. In this
  paper, we investigate the extent to which these new NLG approaches
  profit from the advances in planner expressiveness and
  efficiency. Our findings are mixed.  While modern planners can
  readily handle the search problems that arise in our NLG
  experiments, their overall runtime is often dominated by the
  grounding step they perform as preprocessing.  Furthermore, small
  changes in the structure of a domain can significantly shift the
  balance between search and preprocessing.  Overall, our experiments
  show that the off-the-shelf planners we tested are unusably slow for
  nontrivial NLG problem instances. As a result, we offer our domains
  and experiences as challenges for the planning community.
\end{abstract}

\begin{keywords}
  natural language generation, planning
\end{keywords}


%%% Local Variables: 
%%% mode: latex
%%% TeX-master: "manuscript"
%%% End: 
