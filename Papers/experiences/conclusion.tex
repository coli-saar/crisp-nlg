\section{Conclusion} \label{sec:conclusion}

In this paper, we introduced two novel planning domains arising from
problems in natural language generation: the sentence generation
domain and the GIVE navigation domain. We also reported on the results
of a number of experiments in which we applied off-the-shelf planners
to a set of problem instances in these domains.

Our results were mixed. While modern planners do a pretty good job of
controlling the complexity of search, they also suffer from practical
problems that limit their performance in unexpected ways. In particular, in
both domains the grounding step performed by FF and SGPLAN dominates the
time it takes to perform the search itself. In future work, we hope to
extend our analysis by exploring other planners, such as SHOP2
\cite{DBLP:journals/jair/NauAIKMWY03}, which do not perform the initial
grounding step. We also believe our domains would provide suitable
challenges for planners entered in future editions of the IPC.
%We are also considering to propose some of our domains as IPC benchmarks.

The NLG community's recent interest in planning presents a valuable
opportunity for planning researchers.  While GIVE highlights
planning-related challenges such as plan execution monitoring and plan
presentation (i.e., summarisation and elaboration), it also offers a
platform on which such technologies can be evaluated in experiments with
human users.  Furthermore, more mainstream research on reasoning under
uncertainty, planning with incomplete information and sensing, and resource
management can also be applied in this domain setting. Provided some of the
challenges we have highlighted can be addressed, projects like GIVE (and
other NLG-inspired problem domains) offer a constructive platform for the
planning community to showcase their techniques to a wider audience---and
to improve the quality of their tools for real-world planning tasks. 

%The NLG community's recent interest in planning presents a valuable
%opportunity for planning researchers: provided some of the challenges
%we have highlighted can be addressed, projects like GIVE (and other
%NLG-inspired problem domains) offer a constructive platform for the
%planning community to showcase their techniques to a wider
%audience---and to improve the quality of their tools for real-world
%planning tasks.  Furthermore, GIVE highlights planning-related
%challenges such as plan execution monitoring and plan presentation
%and summarisation and offers a platform on which such technologies can
%be evaluated in experiments with human users.

\section{Acknowledgements}
This work arose in the context of the Planning and Language Interest Group
at the University of Edinburgh. We thank all members of this group,
especially Hector Geffner and Mark Steedman, for interesting discussions.
This work was partially supported by the DFG Research Fellowship ``CRISP:
Efficient integrated realization and microplanning'' and by the European
Commission through the PACO-PLUS project (FP6-2004-IST-4-27657).

%\paragraph{Acknowledgments.} This work is a collaboration that arose
%in the context of the ``planning and language'' group at the
%University of Edinburgh.  We thank all members of this group,
%especially Hector Geffner and Mark Steedman, for interesting
%discussions.  This work was supported by the DFG Research Fellowship
%``CRISP: Efficient integrated realization and microplanning'' and the
%EU project XYZ PACO-PLUS.

%%% Local Variables: 
%%% mode: latex
%%% TeX-master: "experiences"
%%% End: 
