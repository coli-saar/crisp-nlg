\section{Conclusion} \label{sec:conclusion}

In this paper, we introduced two novel planning domains arising from
problems in natural language generation: the sentence generation domain and
the GIVE navigation domain. We also reported on the results of a number of
experiments in which we applied off-the-shelf planners to these domains.
Our main results were mixed. While modern planners do a pretty good job of
controlling the complexity of search, they also suffer from practical
problems that limit their performance in unexpected places. In particular,
in both domains the grounding step performed by both planners dominates the
time it takes to perform the search itself. The NLG community's recent
interest in planning also presents a valuable opportunity for planning
researchers: provided some of the challenges we have highlighted can be
addressed, projects like GIVE offer a constructive platform for the
planning community to showcase their techniques to a wider audience---and
to improve the quality of their tools for real-world planning tasks.

%promising, worthwhile, far-reaching, productive

%In this paper, we have introduced two novel planning domains related
%to problems in natural language generation: the sentence generation
%problem and the GIVE navigation problem.  We have then reported on a
%number of experiments in which we applied off-the-shelf planners to
%these domains.  Our main results were that while modern planners do
%pretty well in controlling the complexity of search, they suffer from
%practical problems that limit their performance in unexpected places.
%In particular, in both domains the grounding step that both planners
%perform tends to take much more time than the search itself.


%%% Local Variables: 
%%% mode: latex
%%% TeX-master: "experiences"
%%% End: 
