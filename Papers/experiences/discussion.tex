\section{Discussion} \label{sec:discussion}

The positive conclusion we can draw from the experiments reported
above is that for the most part, modern planners are pretty good at
controlling the search for the NLG-related inputs that we looked at.
This is particularly true for FF in the sentence generation domain,
which generates nontrivial 14-word sentences in about two seconds.
There is still room for improvement (a special-purpose algorithm for
referring expressions could generate ten of these 14 words in a few
milliseconds), and the search becomes slower -- for obvious reasons --
in the almost completely unconstrained world of Experiment 2, but
search is not the main efficiency bottleneck.

The most restrictive bottleneck in the two domains we have looked at
in this paper is the initial grounding step that almost all modern
planners perform.  Our conclusion from the experiments reported above
is that this is a good strategy in domains where the typical plan
length dominates the typical universe size, i.e.\ where the initial
investment into grounding pays off in saved unifications during the
search.  However, as we have shown, there are natural planning domains
in which relatively short plans must be computed in large universes;
it is not unrealistic for the universe in the sentence generation
domain to consist of thousands of domain individuals and tens of
thousands of actions, some of which take three domain individuals as
parameters.  The common strategy of splitting the universe over
several types of individuals will not be very effective here.

We believe that if planning is to become a mature technology which
real-world users will believe in, it must find a way to deal with the
grounding problem more gracefully than current systems do.  The
research focus in developing planning algorithms has quite rightfully
been on dealing with the search, and for this it is convenient --
depending on the particular algorithm, necessary -- to ground out all
predicates and actions before starting the search itself.  However, we
think that the domains presented here are not that unusual in their
ratio of plan length against universe size.  At the end of the day,
real-world users will care about the \emph{total} runtime of a
planner, and this is more than just the search.

By and large, it is a relatively pleasant experience to come as a
``customer'' to the planning community: Thanks to the competitions, it
is easy to identify and download a fast implementation, at least for
Linux.  However, in the course of our experiments we found (and
reported) bugs in both SGPLAN and FF.  And as our experiments have
shown, there is no one silver bullet: Even planners as closely related
as FF and SGPLAN can differ significantly in their performance on
different domains.

%%% Local Variables: 
%%% mode: latex
%%% TeX-master: "experiences"
%%% End: 
