\documentclass[letterpaper]{article}

\usepackage{aaai} 
\usepackage{times} 
\usepackage{helvet} 
\usepackage{courier} 

\usepackage{graphicx}
\usepackage{url}

\newcommand{\todo}[1]{\textbf{(#1)}}


\title{Experiences with Planning for Natural Language Generation}
\author{Alexander Koller \and Ronald Petrick \\
School of Informatics \\
University of Edinburgh \\
Edinburgh \ EH8 9AB, Scotland, UK \\
\texttt{\{a.koller,r.petrick\}@ed.ac.uk}}

\nocopyright


\begin{document}

\maketitle


\begin{abstract}
We investigate the application of modern planning techniques to domains
arising from problems in natural language generation (NLG). In particular,
we consider two novel NLG-inspired planning problems, the sentence
generation domain and the GIVE (``Generating Instructions in Virtual
Environment'') domain, and investigate the efficiency of FF and SGPLAN in
these domains. We also compare our results against an ad-hoc implementation
of GraphPlan in Java. Our results are mixed. While modern planners are able
to quickly solve many moderately-sized instances of our problems, the
overall planning time is dominated by the grounding step that these
planners perform (rather than search). This has a pronounced effect on our
domains which require relatively short plans but have large universes. We
share our experiences and offer these domains as challenges for the
planning community.  \end{abstract}


\input introduction
\input domains
\input experiments
\input discussion
\input conclusion


\bibliography{bibliography}
\bibliographystyle{aaai}

\end{document}

%%% Local Variables: 
%%% mode: latex
%%% TeX-master: t
%%% End: 
